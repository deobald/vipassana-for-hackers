\documentclass{letter}
\usepackage{hyperref}
\usepackage{xcolor}
\signature{Steven Deobald}
\address{Lot 1 Beaver Harbour Rd. \\ Beaver Harbour \\ Canada B3H 4M1}

\NewDocumentCommand{\codeword}{v}{%
\texttt{\textcolor{blue}{#1}}%
}
\newcommand\hr{\par\vspace{-.5\ht\strutbox}\noindent\hrulefill\par}

\begin{document}
\begin{letter}{Adrianna Tan}
\opening{Dear Adrianna,}

Enclosed is a copy of Charlotte Joko Beck's ``Nothing Special''.
I do not endorse her teaching or her particular brand of Buddhism (nor any other) but, immediately following my second 10-day meditation course, I found her writing to be a great help in untangling the confusion which surfaced from my experience.
Perhaps you will also find it similarly useful.
It has little or nothing to do with the following letter.

In the past, I've written essays in lieu of letters (or emails) to individuals.
This time, I thought I might write and publish the letter instead.
Let me know what you think about that idea.

I write to you to speak, ostensibly, about gurus.
Starting with an essay at first, I penned a title of ``On Gurudom'' which caused the immediate \codeword{rm} of that file.
Such an essay telegraphs its conclusion loudly and obnoxiously: ``Gurus Bad!''
That's not my intention.
There's also a lot of periphery to the teachings and their respective teachers that don't have anything to do with gurudom, per se.

It's impossible for me to write something for every economy, cultural context, linguistic background, and ambient religion.
Instead, I'll try my best to relate my personal experiences in a way that might be appreciated globally -- as a story, if nothing else.

\hr

There were no gurus in my childhood.
By the time I was old enough to appreciate what religion and spirituality meant to the adults in my community,

\end{letter}
\end{document}
