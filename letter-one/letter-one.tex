\documentclass{letter}
\usepackage{hyperref}
\usepackage{xcolor}
\signature{Steven Deobald}
\address{Lot 1 Beaver Harbour Rd. \\ Beaver Harbour, NS \\ Canada B3H 4M1}

\NewDocumentCommand{\codeword}{v}{%
\texttt{\textcolor{blue}{#1}}%
}
\newcommand\hr{\par\vspace{-.5\ht\strutbox}\noindent\hrulefill\par}

\begin{document}
\begin{letter}{Adrianna Tan}
\opening{Dear Adrianna,}

Enclosed is a copy of Charlotte Joko Beck's \textit{Nothing Special}.
I do not endorse her teaching or her particular brand of Buddhism (nor any other) but, immediately following my second 10-day meditation course, I found her writing to be a great help in untangling the confusion which surfaced from my experience.
Perhaps you will also find it similarly useful.
It has little or nothing to do with the following letter.

In the past, I've written essays in lieu of letters (or emails) to individuals.
This time, I thought I might do the opposite and write (then publish) the letter instead.
Let me know what you think about that idea.

I write to you to speak, ostensibly, about gurus.
Starting with an essay at first, I penned a title of ``On Gurudom'' which caused the immediate \codeword{rm} of that file.
Such an essay telegraphs its conclusion loudly and obnoxiously: ``Gurus Bad!''
That's not my intention.
There's also a lot of periphery to the teachings and their respective teachers that don't have anything to do with gurudom, per se.

It's impossible for me to write something for every economy, cultural context, linguistic background, and ambient religion.
Instead, I'll try my best to relate my personal experiences in a way that might be appreciated betwixt our two homelands -- as a story, if nothing else.

\hr

There were no gurus in my childhood.
By the time I was old enough to appreciate what religion and spirituality meant to the adults in my community, the various sects of Canadian Christianity (the only spiritual practice of note in a town with a population of 1000) had already begun to collapse.
From most churches, gone were the severe priests and pastors of an immigrant generation.
The death rattle of podunk churches sounded like soft rock on Friday nights amongst a congregation who now referred to their minister by her first name.
At that age, I never thought of church leaders as \textit{teachers}, even to those they preached to.
Certainly not to me.

When I moved to India for the first time in 2006, I was a drunken wreck and no one in my life would mistake the year I spent in Pune for a trip of self-discovery.
But when I moved to Bangalore in 2011, the question was surprisingly common: ``are you going to find a guru? to start a practice? to go on a spiritual quest?'' ... ``did you find yourself there?''
The truth is, Bangalore was just another escape.
I hated my job in high finance and, what the heck, maybe I could teach some kids Clojure for a year before finding another real job.
Had I known in advance that I'd spend nearly a decade in India, I probably never would have gone.

I did take two meditation courses in India, but they were Burmese (Vipassana) and Japanese (Zazen), and bore little in common with meditation schools one would normally associate with India.
At the time, my Indian friends found the long-bearded God Men of India as offensive as I might find the minister of a new age megachurch in the USA.
I found them cartoonish.

Leaving aside the post-Osho beards and robes (which many lower-and-middle-class Indians don't identify as red flags the way you and I might), I found it only took a few seconds to search for ``XYZ net worth'' or ``XYZ controversy'' and justify my hesitations.
These were clearly businessmen.
Most didn't bother hiding it.

Over the years, I began to navigate and solidify my own meditation practice.
As I did this, I encountered a number of different schools, practices, ideas, philosophies, teachers, and students.
Curiously, what I found was that the subtler schools and teachers were often the most dangerous.

\hr

The primary difficulty in finding a teacher, as you had mentioned in our conversation, is the power dynamic.
This is unavoidable.
And it is for this reason that I generally discourage my friends from seeking a teacher at all.
Secondarily, not only is the power dynamic a problem, but any significant meditation practice will also severely alter the practitioner's mental state from time to time.
This exacerbates the problem of power dynamics between teacher and student: it is when the student is vulnerable that she needs the teacher's help the most.

Without contradiction, I also discourage people from self-study.
All but the lightest (MBSR and other common forms of mindfulness, for example) meditation practices are bound to place the practitioner into deeply troubled psychic states sooner or later.
Alone in our living rooms, listening to pre-recorded instructions in an app without guidance or a safety net, is not the time to dig up our most penetrating unconscious psychological traumas.
I've heard too many horror stories of dabbling students painting themselves into a corner to believe that exclusive self-study is a safe approach for anyone.

Instead, I have found that seeking out \textit{systems}, rather than gurus or my hubris-fuelled insistence on self-reliance, to be the most effective path.
Along the way, I have encountered teachers, of course -- but these teachers tended to be litmus tests for selecting a system, rather than the other way around.
A teacher is a representative of her chosen practice.

\hr

With that in mind, I'll return to the idea of subtle dangers for a moment.
Outside of the more obvious warning signs (``this lady is teaching X to make money'' or ``this dude is in jail for rape -- maybe his school isn't for me''), I have found gross-but-less-obvious warning signs over the years.

The first is touching.
It seems very strange to people accustomed to a lot of physical contact that so much as a handshake should be forbidden between teacher and student (or between the students themselves, in a practice environment).
For some, a rule like ``no touching'' conjures up images of George Bluth Senior shouting in the jail in the first season of \textit{Arrested Development}.
Still, the fact that ``no touching'' presents itself as a binary rule makes it very easy to adhere to, without wading into an extreme, such as forced chastity, which is known to have the opposite effect.
There are any number of problematic events, such as the stripping of W.S. Merwin and Dana Naone by Chögyam Trungpa, which are obviated by such a rule.
Though the Halloween party of 1975 is something we'd consider unforgiveable today, it may have been forgiven by all involved back in the 70s.
There are new grey areas of abuse in this decade, but ``no touching'' (if it is adhered to), at least eliminates direct physical abuse.
I attended a week-long meditation course once which encouraged the students to ``greet each other'' (with hugs) at the point of course completion.
It was innocent, but I still found that much physical contact in a meditation environment a rather shocking experience.

The second is alcohol.
Abuses of power, both physical and psychological, are given unnecessary fuel by alcohol.
For a long time, I was drawn to Zen.
However, even with all my other hesitations set aside about Zen practice, I could never get over the use of alcohol -- although I have been sober for 8 years I still acutely remember the damage it does to the human decision-making process.
Where there is alcohol, there is bound to be abuse of power, eventually.

The last is money.
There are a number of East-Meets-West schools and practices which intentionally adopt a commercial position.
Much like cartoonish God Men, meditation schools and apps which make no bones about their for-profit nature don't seem as dangerous to me as those which sit somewhere in the middle.
It is unlikely that anyone will learn meditation on the floor of an American yoga studio, much less harm themselves psychically while doing so.
The danger comes when a school presents itself as a spiritual outlet -- a temple, a monastery, or a meditation centre.
In these locations, if participation is transactional, the lines become blurred and there is a risk that students will be exploited.
Many meditation schools teach for free, on a donation basis.
Although I have personally attended meditation courses which cost money, and although they were often useful and informative, ultimately none of them were for me.

The subtle dangers are hinted at by the danger of money.
From the God Men of India to the spandex spirituality of the USA to stoic philosophies of Europe to African Cosmology to Buddhism in Southeast Asia... there's a lot out there.
For better or worse, it's all getting a bit blended up now.
The subtlety comes in situations where a path, school, or teacher is not just \textit{not obviously dangerous} but perhaps seems particularly safe or inviting.
As we become increasingly invested in any particular path, the dangers present on that path (and they're on every path, obviously) are amplified ... we always find it harder to divest ourselves of something we've sacrificed for.

Although it would be presumptuous of me to assume you're specifically and narrowly interested in the teachings of the Buddha, I do think the foundational advice he provides is a safe starting position (for me, at least).
To paraphrase, in one sutra he suggests to students: ``Inspect my teaching like a golden trinket you might buy in the market. Try to find flaws in what I am teaching you.''
In others, he insists on relying on ``yathā-bhūta'' or ``seeing things as they are, with direct experience, devoid of any concepts''.
As soon as I have found myself relying on someone -- or some belief -- I have experienced the aforementioned sacrifice, which is a dangerous commitment even in small doses.
On the occasions when I place my trust in myself, my intuition, and my skepticism, I have managed to avoid these dangers.

\hr

Regarding intuition, and the teachers I have studied under, the litmus test I spoke of earlier has always materialized only by chance.
The conditions for such a test cannot be created at will; we have to be patient.

In one such example, a minor irritant (a malfunctioning microphone) managed to push the teacher to a point of visible frustration with her co-teacher and her partner (who was volunteering).
It was surprising to me how quickly my opinion of that teacher, and the practice she taught, was shattered by observing her reaction to such an insignificant stimulus.

In another pair of examples, one teacher was verbally assaulted by a student during a Q\&A period and another had to deal with a massive snow storm and power outage.
In the first instance, the student accused the teacher of being a fraud, essentially, and then demanded the teacher (who was not Japanese) translate some Japanese words for him.
The teacher was clearly shocked by the interaction... but didn't respond with irritation or frustration.
Instead, he smiled and responded to the student patiently.
During the snow storm, the teachers not only maintained their composure under difficult circumstances, but continued to support the course volunteers and coordinate safety measures, day after day, for students.

No single event (or response to an event) could convince me of the benefits a particular practice has conveyed (or not) onto a teacher.
Still, every observation you make about your teachers will paint pictures of them for you -- and I think our own intuition can tell us a lot about the people we interact with in this world.

\hr

And so I wish you the best of luck in your pursuit of a system and teacher.
I hope you try many systems and practices.
I hope you begin with a breadth-first search, if you can.
I hope you can avoid commitment until you are certain.

When it comes to meditation, there are many blind alleys.
Still, there is no other aspect of my life as precious as meditation.
It is the greatest thing I know.
The search is all made worth it when you find a practice that really changes you.

\closing{Sincerely,}


\end{letter}
\end{document}
