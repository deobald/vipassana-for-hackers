\documentclass{article}

\usepackage{enumitem}

\setlength{\parindent}{1.3em}
\setlength{\parskip}{0.7em}
\setlist{noitemsep}

\begin{document}

\begin{titlepage}
   \vspace*{\stretch{1.0}}
   \begin{center}
     \Huge\textbf{Vipassana for Hackers}\\
     \Huge{Paper Three: Why Meditate?}\\
     \vspace{5cm}
     \large\textit{Steven Deobald}\\
     \large\textit{Version 0.1 DRAFT}\\
     \large\textit\today\\
     \vspace{5cm}
     \large\textit{}\\
   \end{center}
   \vspace*{\stretch{2.0}}
\end{titlepage}

\begin{center}
  \Huge{Target Audience}
\end{center}

\textit{Vipassana for Hackers, Paper One: Curious Mechanics} was written to avoid discussing the
outcomes or consequences of meditation in detail. The focus of that paper was only
the internal mechanics of Vipassana meditation, to pique the interest of potential
meditators who had heard of Vipassana elsewhere. Outcomes are discussed only so far
as they assist the reader in understanding what is written earlier in the paper
regarding the senses. \textit{Paper Two: The Brain} goes further into the mechanics as they pertain to the
nervous system. Here, outcomes are discussed as they pertain to
neuroplasticity. Neither paper directly discusses why an individual might choose to
try this particular technique of meditation.

As before, the ``Hacker'' of \textit{Vipassana for Hackers} is not meant to identify
computer programmers. Instead, it is meant as a label for a culture of curious and
creative people who enjoy exploring, learning, and creating.

\textit{Paper Three: Why Meditate?} is written for anyone who has ever asked
themselves that very question or asked that question of their friends who
meditate. It is for both those who are curious about the practice of Vipassana
specifically and those who are curious about meditation in general. It is for people
who have meditated in other traditions and are curious about the benefits of
Vipassana. It is also for people who have never meditated in their entire lives. It
is intended for anyone who keeps hearing about Vipassana meditation --- in the media,
in books, and from friends --- and wants to learn what all the fuss is about.

The reader need not have read \textit{Paper One} or \textit{Paper Two}. In fact, it
is the intention of this paper to be the most accessible of the series and readers
with only a faint interest in the topic of meditation should start here.

\pagebreak

\begin{center}
  \Huge{Vipassana Basics}
\end{center}

Before we get to a discussion about why meditation is valuable, some basic
understanding of what meditation is (and isn't) is required.

The technique of Vipassana is based on a single underlying principle:

\Huge\textbf{Every experience which emerges in the mind, whether a thought or emotion, always surfaces
with a corresponding sensation on the body.}

It is important to understand this point as it underpins all other aspects of the
technique of Vipassana. Someone who is learning Vipassana need not accept this
principle as fact. Rather, a 10-day Vipassana course is a sort of laboratory where the
principle can be tested and experienced for oneself.

TODO: sense circle diagram

5 Senses by Daniel Falk from the Noun Project: https://thenounproject.com/daniel2021/collection/human-body-senses/
Thought by Nociconist from the Noun Project: https://thenounproject.com/search/?q=thought&i=2025873
Heart by Rafael Garcia Motta from the Noun Project: https://thenounproject.com/search/?q=heart&i=807960
Body by Makarenko Andrey from the Noun Project: https://thenounproject.com/search/?q=body&i=789989

\begin{itemize}
  \item posture
  \item sleep
  \item health (activation / motivation)
  \item ethics (activation / motivation)
  \item your children: a. knowing how to meditate, b. cross-legged posture
  \item emotion (i.e. anger)
  \item mundane sphere / productivity (21 lessons, seinfeld)
  \item reset frame of reference outside oneself, outside one's own lifetime: ``trees for god'' and obvious karma (sidu/booga smoking)
  \item unlearning obsessive / repetitive thought, enhancing creativity
  \item controlling unbounded sexuality without repression
  \item clarity: in thought, work, planning
  \item clarification: ``isn't that what makes us human?'' (emotions) --- rather, what makes us animal
  \item Die Standing Up

\end{itemize}


\end{document}
