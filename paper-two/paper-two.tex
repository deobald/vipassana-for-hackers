\documentclass[twocolumn]{article}

\usepackage{polyglossia}
\setdefaultlanguage{english}
\setotherlanguage{sanskrit}
\usepackage{fontspec}
\newfontfamily\devanagarifont[Scale=MatchUppercase]{Kohinoor Devanagari}

\usepackage{graphicx}
\usepackage{subcaption}
\usepackage{float}
\usepackage[bottom]{footmisc}
\usepackage{enumitem}
\usepackage{hyperref}

\setlength{\parindent}{1.3em}
\setlength{\parskip}{0.7em}
\setlist{noitemsep}

\begin{document}

\begin{titlepage}
   \vspace*{\stretch{1.0}}
   \begin{center}
     \Huge\textbf{Vipassana for Hackers}\\
     \Huge{Paper Two: The Brain}\\
     \vspace{5cm}
     \large\textit{Preethi Govindarajan}\\
     \large\textit{Version 0.1}\\
     \large\textit\today\\
     \vspace{5cm}
     \large\textit{Contributions by: Steven Deobald}\\
   \end{center}
   \vspace*{\stretch{2.0}}
\end{titlepage}

\begin{center}
  \Huge{Context}
\end{center}

I took my first Vipassana course in 2017. Vipassana meditation was so unlike anything
I had ever experienced before, after the course I was left extremely curious
about what exactly had happened to me during those ten days. For months afterward, I
spent my mornings and evenings wading through the research in the field of
meditation, with a specific focus on white papers dealing with the effects a
10-day Vipassana course has on the brains of participants. The research in this area
is limited. The quality research which does exist usually uses a sample of highly
experienced meditators rather than beginners and/or self-reports rather than
objective measures.

Over the past year, I have tried to write down what I experienced during that first
10-day course and the 10-day courses I have taken since, corroborating my experience
with research that does exist regarding meditation and the brain.

\textbf{Proviso:} S.N. Goenka, the principal teacher of Vipassana meditation,
actively dissuades students from precisely the sort of brain-centred biological
inquiry presented in this paper.

\begin{quote}
  \textbf{The brain itself is just a physical organ. As you deal with other parts of
    the body, you deal with the brain in the same way, that's all. Nothing special to
    do with the brain. But the mind is totally different. In the West, all importance
    is given to the brain as if the mind is located here. Nothing doing, it is
    everywhere. The mind is in the whole body. So give attention to the whole
    body.} \cite{goenkabrain}
\end{quote}

This paper does not contradict Mr. Goenka's sentiment, but instead acts as a starting
point for those who view the brain as the quintessential physical component of mind.

\textbf{Disclaimer:} Although my primary field of research is biology, my work is far removed
from neuroscience. I have tried to simplify the research available so as to better
understand it. If there are any corrections to make or editing in terms of the
content, please feel free to get in touch with me.

\begin{center}
  \Huge{Brain Function and Anatomy}
\end{center}

Before dissecting the experience of meditation as it pertains to the brain, this
section describes the different parts of the brain that have shown up in
scientific literature as correlated to meditative practices. Neuroscience divides the
brain into the reified geography of the brain, its anatomy, and the abstract concepts
governed by brain activity, its function. Between concrete physiology and abstract
outcomes exist networks of cooperative structures which correspond to general
high-level activities of the brain.



Default Mode Network:
These are a set of structures in the mid-line of the brain (which is a evolutionarily more primitive area related to memory and emotion) which connects to structures in the cortex (which is evolutionarily the most recent part of the brain and contains the executive and higher order functions). It is the constellation of regions that light up when the brain is not engaged in any external/goal-oriented tasks (1). These regions are functionally connected and are involved in
a) The neurological basis of the self —
The regions involved in this are the Posterior Cingulate Cortex (PCC), venteromedial Prefrontal Cortex (vmPFC) and the Inferiour Parietal Lobule (IPL) (1).
Cerebral blood flow and metabolic rate is 40\% higher in the PCC compared to the average across the brain making it one of the most active regions of the brain. (Leech R, Sharp DJ (July 2013). “The role of the posterior cingulate cortex in cognition and disease”. Brain. 137 (Pt 1): 12–32.)
The PCC has been thought to mediate interactions between emotion and memory. It consistently lights up when a person recollects something from their life and depending on the emotional importance of the memory the strength of the lighting up changes. It is also activated by emotional stimuli both positive and negative and acts as an interface between the world and us by gauging the importance of different stimuli(2).

Magnetic resonance Image of the areas of the brain in the Default Mode Network. http://www.frontiersin.org/Neurotrauma/10.3389/fneur.2013.00016/full
The mPFC is divided into venteromedial PreFrontal Cortex (vmPFC) and dorsomedial Prefrontal Cortex (dmPFC). The vmPFC is involved in creating a conceptual self. It is involved in self-related processing and assigns personal significance to self-related information (3).
It is hypothesized that the activation of these three regions of the brain are responsible for providing a sense of self — as a subjective agent in space and time. It functions as a network in which phenomena such as self representations are accessible via the PCC and are gated into conscious awareness by activity in the MPFC which is influenced by changing internal and external demands. The PCC which acts as a brain-wide connectivity hub through which a unitary notion of self is created by a large scale integration of DMN activity (4).
b) Thinking about others —
The Dorsal Medial Prefrontal Cortex (dmPFC) is involved in Theory of the mind(considering the mental states of others), empathy, moral reasoning and altruistic behavior (1,5,6)
c) Autobiographical memory and future simulations —
The DMN is also involved in remembering the past and imagining the future and story comprehension(1).
Salience Network (SN):
Salience: The perceptual quality by which an observable thing stands out relative to its environment
The SN is an intrinsically connected large-scale network anchored in the Anterior Insula (AI) and dorsal Anterior Cingulate Cortex. Both these regions have reached a high degree of specialization in the great apes. It is the collection of the regions in the brain that help decide which stimuli deserves our attention. It acts as a switch between the internally directed DMN and the externally directed Central Executive Network (involved in high level cognitive functions) (7).

The Salience Network consists of the Anterior Cingulate Cortex and the Insula. It helps to switch between the Default Mode Network and the Central Executive Network
a) Anterior Cingulate Cortex (ACC): The ACC is the front end of the cingulate cortex and collars around the corpus callosum (the band connecting the two hemispheres of the brain). It is the connector between the emotional (limbic system) and the cognitive part of the brain. It is involved in functions such as attention allocation, reward anticipation, decision making, morality, impulse control, emotional awareness and registering pain (8–12). It also appears to play a role in the regulation of Autonomic functions such as blood pressure and heart rate (13).
b) Anterior Insular Cortex (AIC): The insula is a part of the cerebral cortex that is located deep within the sulcus (which is the fissure separating the four lobes of the brain). The AIC projects itself into the amygdala. It is involved in multimodal sensory processing such as audio-visual integration tasks, interoceptive awareness (so its activity is directly related to ones sense of internal body states), empathy and conscious awareness (14). It also plays a role in the regulation of autonomic functions such as bodily sensations (as well as where the degree of pain is judged), taste and control of the immune system (15).
The AIC and ACC together give rise to our interoceptive and conscious self-awareness.
Central Executive Network:
This network is activated when a high-level cognitive tasks or external goal-oriented tasks are being performed. Executive functions are cognitive processes involved in cognitive control of behavior.
The regions of the brain involved in this are Dorsolateral Prefrontal Cortex, Orbito Frontal Cortex and the Posterior Parietal Cortex.
Dorsolateral Prefrontal Cortex (DL-PFC): It is a part of the pre-frontal cortex and is found in primates and humans. It is a part of the Central Executive Network. The DL-PFC is involved in higher cognitive processes including working memory (holding different pieces of information and manipulating and using them for tasks), selective attention, cognitive flexibility (switching between tasks) and planning. It also seems to be involved in social cognition and lying and increases dopamine levels in the brain.
Barbey AK, Koenigs M, Grafman J (May 2013). “Dorsolateral prefrontal contributions to human working memory” (PDF). Cortex. 49 (5): 1195–1205.
Monsell S (2003). “Task switching”. Trends in Cognitive Sciences. 7 (3): 134–140. doi:10.1016/S1364–6613(03)00028–7. PMID 12639695.
Ito, Ayahito; Abe, Nobuhito; Fujii, Toshikatsu; Hayashi, Akiko; Ueno, Aya; Mugikura, Shunji; Takahashi, Shoki; Mori, Etsuro (2012). “The contribution of the dorsolateral prefrontal cortex to the preparation for deception and truth-telling”. Brain Research. 1464: 43–52. doi:10.1016/j.brainres.2012.05.004
Orbito Frontal Cortex (OFC): This is an area found in front of both the hemispheres and just above the eye. It is a part of the Pre-Frontal Cortex and it is thought to be involved in the cognitive process of decision making. It is thought to represent emotion and reward in decision making. It also receives input from multiple sensory modalities and in turn activates the amygdala and the hypothalamus
Fuster, J.M. The Prefrontal Cortex, (Raven Press, New York, 1997
Rolls, ET (November 2004). “Convergence of sensory systems in the orbitofrontal cortex in primates and brain design for emotion”. The Anatomical Record Part A: Discoveries in Molecular, Cellular, and Evolutionary Biology. 281 (1): 1212–25. doi:10.1002/ar.a.20126
Price, Joseph L. (2006). “Chapter 3: Connections of the orbital cortex”. In Rauch, Scott L.; Zald, David H. (eds.). The Orbitofrontal Cortex. New York: Oxford University Press. p. 45.
Posterior Parietal Cortex (PPC): It receives information from the Auditory, Visual and Somatosensory systems and in turn activates the motor cortex or DL-PFC.
Martin, R. E. (n.d.). Let’s Get to Know the Parietal Lobes! [PDF]. Retrieved from \url{http://gablab.mit.edu/downloads/Parietal_Primer.pd}

Approximate location of the OFC on an MRI
(Maybe remove)
The limbic system: The limbic system is a complex set of structures that lies on both sides of the thalamus, just under the cerebrum. It includes the hypothalamus, the hippocampus, the amygdala, and several other nearby areas. It appears to be primarily responsible for our emotional life, and has a lot to do with the formation of memories.

The course
I applied for this ten day course after I quit my job. I was very nervous as soon as the acceptance came through until the first day. I had never done anything like this before and I was very comfortably placed within the idea that meditation with its spiritual connotation was a waste of my time. It was for people who had time to waste and I was not one of them. I went for my first course in Chennai and once I got there I had to fill all the forms all over again keep all my luggage in a locker and go to my room. I had to share the room with one other person and I was so out of my element and knew everyone could see me and tell that I was the odd one out, the one that was not supposed to be there and I was too scared to actually make conversation with anyone before silence was enforced. So with my brain filled with thoughts about myself, about the people around me, I started this meditation business. Within the first two days, I had made up elaborate stories about my fellow meditators, all superheroes tirelessly working to save humanity together. My default mode network was in overdrive.
For the first 3.5 days you are asked to focus your attention on the area below your nostrils and above your upper lip. You start with observing your breath coming in and going out. As the teacher mentions in the late evening discourses, the reason for starting with your breath is because this activity is both conscious and unconscious and acts as a bridge.
At least during the meditation hours, I was trying hard, I wanted to give this an honest shot.I focused on the small patch beneath my nostrils above my upper lip. To begin with when you are asked to observe your breath and your mind wanders, that is the default mode network working, as you bring your attention back to your breath, your salience network is activated. Slowly at first, with your mind wandering for many many minutes before you realize you are not with your breath but within the 3.5 days, it pretty quickly learns to bring it back and almost does it by habit. As I gave in to this activity, tried to focused more, tried to sit still, there were longer periods of awareness on that patch of skin and stranger sensations arising and passing. By the third day, it felt like entire ecosystems writhing and flopping and crashing were alive in the area below my nostrils above my upper lip. There have been studies that have shown that once you bring your attention back to breath, the neural structures involved in the control of the Autonomic Nervous System and attention start firing more actively (DL-PFC-a part of the CEN, the Hippocampus and the ACC — part of the SN). There are also global dampening changes seen in the brain especially within the DMN and the Amygdala (which is involved in the flight-or-fight response/stress response). These changes have antagonistically been termed the relaxation response. This relaxation response can be thought of as a gateway to altered states of mind. (16)
There has also been research in mice, showing a cluster of nerves called the pre-botzinger complex found in the brain stem of most mammals which fire with every breath taken. This breathing pacemaker seems to work for all kinds of breathing such as yawns, sighs and gasps. They also appear to play a role in calming and arousal (17). It would be interesting to see the activation in these neuron clusters during the first three days of the course even amongst novice meditators.
After lunch on the fourth day. The instructions for Vipassana are given, where you transfer your attention from the patch below your nostril to the top of your head. From there, for over two hours, the instructions are given to slowly move your attention till your feet and then repeat. The two criteria for observation here are focused attention and open monitoring (observing sensations objectively).
The Salience Network (ACC and AIC) and the Central Executive Network (DL-PFC, PPC and OFC) which are involved in the control of attention start firing rapidly. Usually as the person is engaged in an external task, the Central executive network (Orbito Frontal Cortex and Posterior Parietal Cortex), receiving signals from all the sensory networks (Visual, Audio, Somatosensory etc..) continuously and in response sends signals to the different motor cortices, amygdala, hypothalamus to perform a task, feel an emotion, etc..
Now imagine a situation where once the Salience Network is activated and in turn activates the Central Executive Network and the task that it is supposed to perform is not external but to observe sensation and do nothing. So as you sit there with your eyes closed and not moving and scanning your body, you are doing what the central executive network has always been doing except attempting to not react.
This is hard at first, with the body involuntarily reacting by jerking and writhing in response to your observation. Your awareness itself is jerky and it is hard to observe sensations consciously.
There is also the pain. This eclipses all other sensations. It is difficult trying to be aware of any other sensation with the pain that was emanating from my legs and back. The Orbito Prefrontal Cortex (OFC), is involved in assigning emotion to sensation (18) and the ACC plays a vital role in perception of sensation. These two parts of the brain are usually functionally connected. In the beginning, as I sat trying to observe sensations and I was only aware of the pain, the ACC was firing rapidly and the OFC then assigned these signals to the limbic system (including the amygdala and hippocampus) which then put me in a less than ideal emotional state.
But as your awareness gets smoother, this functional connectivity between the ACC and the OFC slowly starts changing and as you continue to pay attention to the different sensations and attempt to not assign emotion and observe with a non-judgmental attitude particularly toward unpleasant stimuli, the OFC still fires but the limbic system is quietened, so you react less. It is like both your conscious and unconscious awarenesses are slowly aligning until some points when it feels like it is pure awareness moving through your body or some parts of your body. In addition the insula (part of the Salience Network) and the somatosensory cortex which are involved in sensory awareness were firing more actively.
(The somatosensory cortex is located in the gyrus and the parietal lobes: The sensory homunculus which is a part of the somatosensory cortex is a cortical representation of the body based on the degree of sensory innervation)

Somatosensory and Motor cortices
Another strange thing that starts happening is as your mind wandering slows down and you sit with your eyes closed, your sense of your self as a subjective agent in space and time changes, first just very slightly, as you close your eyes and meditate, it feels like you have shifted in space or turned in different directions (during these times the vmPFC and PCC which are a part of the DMN become less active), and then as the meditations get deeper and sometimes, the DMN quietened down completely for a very short duration and in addition to this, the insula is firing more rapidly, it could lead to other alternative networks of consciousness arising. It is as if losing the influence of one’s narrative leads to new insights that are usually kept from consciousness. Experiencing this even for a few seconds can lead to lasting changes even after this activity wears off. These experiences can be quite similar to being on certain kinds of drugs or even in some cases of ecstatic seizures where the abnormal activity of the anterior insula leads to heightened self awareness and feelings of bliss and lack of ambiguity. (Fabienne Picard, State of belief, subjective certainty and bliss as a product of cortical dysfunction, Cortex, Volume 49, Issue 9, 2013,
Pages 2494–2500ISSN 0010 9452, https://doi.org/10.1016/j.cortex.2013.01.006.)
So to conclude, I believe attending a ten day course literally did change the habit pattern of my mind at least for a while.
It quietens the Default Mode Network and causes the Salience Network and certain parts of the Central Executive Network to fire more rapidly (the parts that are aware of the sensations while quietening the reacting part of this network). This causes changes to a persons behavior at least for a short while and this is usually seen as soon as a person leaves the course. But repeated practice is essential for any sustained neuroplasticity since rewriting many years of habit formation requires more than a ten-day course. However, the ten day course is well designed to give the minimum observational change neccesary for people interested to keep practicing.
There have also been studies looking at the structural changes in the brain and functional changes especially to the default mode network and the salience network but these are usually measured after one has clocked over thousands of hours of practice. Therefore I am not delving into those right now.
How is this different from just concentrating on tasks hard? It is different because when concentrating on any task, your default mode network is quetened and the salience network and CEN are firing entering you into a state of flow but the reacting part of the brain is also working and working hard.
There are aspects to this technique that I have not understood or experienced that I have not delved into right now and may do once i know it better.
I was really intrigued as I left the center as to what had occurred over those ten days that changed my brain, I very quickly noticed though that as soon as talking was allowed, these changes slipped away really quickly, but with each hour long sitting a small tiny bit sticks just a little more.

\pagebreak


References

Andrews-Hanna, Jessica R. (2012–06–01). “The brain’s default network and its adaptive role in internal mentation”. The Neuroscientist: A Review Journal Bringing Neurobiology, Neurology and Psychiatry. 18 (3): 251–270. doi:10.1177/1073858411403316. ISSN 1089–4098. PMC 3553600. PMID 21677128.
Maddock, Richard J.; Garrett, Amy S.; Buonocore, Michael H. (January 2003). “Posterior cingulate cortex activation by emotional words: fMRI evidence from a valence decision task”. Human Brain Mapping. 18 (1): 30–41.
Andrews-Hanna, Jessica R.; Smallwood, Jonathan; Spreng, R. Nathan (2014–05–01). “The default network and self-generated thought: component processes, dynamic control, and clinical relevance”. Annals of the New York Academy of Sciences. 1316(1): 29–52. doi:10.1111/nyas.12360. ISSN 1749–6632. PMC 4039623. PMID 24502540.
Davey CG, Pujol J, Harrison BJ. Mapping the self in the brain’s default mode network. Neuroimage. 2016 May 15;132:390–397.
Isoda, M., \& Noritake, A. (2015). What makes the dorsomedial frontal cortex active during reading the mental states of others?. Neural basis of social learning, social deciding, and other-regarding preferences, 51.
Waytz, A., Zaki, J., \& Mitchell, J. P. (2012). Response of dorsomedial prefrontal cortex predicts altruistic behavior. The Journal of Neuroscience, 32(22), 7646–7650.
Menon, V; Toga, A (2015). Salience Network. Elsevier. pp. 597–611. ISBN 978–0–12–397316–0.
Pardo JV, Pardo PJ, Janer KW, Raichle ME (January 1990). “The anterior cingulate cortex mediates processing selection in the Stroop attentional conflict paradigm”. Proceedings of the National Academy of Sciences of the United States of America. 87 (1): 256–9. doi:10.1073/pnas.87.1.256. PMID 2296583.
Bush G, Vogt BA, Holmes J, Dale AM, Greve D, Jenike MA, Rosen BR (January 2002). “Dorsal anterior cingulate cortex: a role in reward-based decision making”. Proceedings of the National Academy of Sciences of the United States of America. 99 (1): 523–8. doi:10.1073/pnas.012470999. PMC 117593. PMID 11756669.
Sevinc G, Gurvit H, Spreng RN (July 2017). “Salience network engagement with the detection of morally laden information”. Social Cognitive and Affective Neuroscience. 12 (7): 1118–1127. doi:10.1093/scan/nsx035. PMID 28338944.
Jackson PL, Brunet E, Meltzoff AN, Decety J (2006). “Empathy examined through the neural mechanisms involved in imagining how I feel versus how you feel pain”. Neuropsychologia. 44 (5): 752–61. doi:10.1016/j.neuropsychologia.2005.07.015. PMID 16140345.
Bush G, Luu P, Posner MI (June 2000). “Cognitive and emotional influences in anterior cingulate cortex”. Trends in Cognitive Sciences. 4 (6): 215–222. doi:10.1016/S1364–6613(00)01483–2. PMID 10827444
Gianaros PJ, Derbyshire SW, May JC, Siegle GJ, Gamalo MA, Jennings JR. Anterior cingulate activity correlates with blood pressure during stress. Psychophysiology. 2005;42(6):627–35.
Xiaosi Gu, Patrick R. Hof, Karl J. Friston, Jin Fan. J Comp Neurol. 2013 Oct 15; 521(15): 3371–3388.
Rolls ET. Functions of the anterior insula in taste, autonomic, and related functions. Brain Cogn. 2016 Dec;110:4–19.
Lazar SW; Bush George; Gollub RL.; Fricchione, GL.; Khalsa G; Benson H. Functional brain mapping of the relaxation response and meditation.
Neuroreport. 11(7):1581–1585, May 15, 2000.
J. Muñoz-Ortiz, E. Muñoz-Ortiz, L. López-Meraz, L. Beltran-Parrazal, C. Morgado-Valle. The pre-Bötzinger complex: Generation and modulation of respiratory rhythm, Neurología (English Edition), 2018, ISSN 2173–5808,
https://doi.org/10.1016/j.nrleng.2018.05.006.
REMPEL‐CLOWER, N. L. (2007), Role of Orbitofrontal Cortex Connections in
Emotion. Annals of the New York Academy of Sciences, 1121:
72–86. doi:10.1196/annals.1401.026


\pagebreak

\begin{thebibliography}{9}
\raggedright

\bibitem{goenkabrain}
  S.N. Goenka.
  \url{https://www.vridhamma.org/A-store-house-of-answers-by-Shri-S-N-Goenka}
  \textit{Answers by Mr. S. N. Goenka}

\bibitem{thenewsroom}
  Aaron Sorkin.
  \url{https://www.youtube.com/watch?v=2C6h-Yyx9Yk}
  \textit{The Newsroom}.
  HBO, 2012.

\end{thebibliography}

\end{document}
