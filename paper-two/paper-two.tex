\documentclass[a4paper, amsfonts, amssymb, amsmath, reprint, showkeys, nofootinbib, twoside]{revtex4-1}
\usepackage[english]{babel}
\usepackage[utf8]{inputenc}
\usepackage[colorinlistoftodos, color=green!40, prependcaption]{todonotes}
\usepackage{graphicx}
\usepackage{subcaption}
\usepackage{float}
\usepackage[bottom]{footmisc}
\usepackage{enumitem}
\usepackage{hyperref}

\bibliographystyle{apsrev4-1}
\setlist{noitemsep}

\begin{document}

\title{%
  \Huge{Vipassana for Hackers} \\
  \large{Paper Two: The Brain} \\
  \large\textit{Version 0.2}
}
\author{Preethi Govindarajan}
\email[Correspondence email address: ]{preethi@deobald.ca}
\affiliation{Siggu.org}
\date{\today}

\begin{abstract}
  \todo{add an abstract} This is the abstract. etc. etc.
\end{abstract}

\keywords{neuroscience, vipassana, meditation}

\maketitle

\listoftodos

\section{Introduction}

I took my first Vipassana course in 2017. Vipassana meditation was so unlike anything
I had ever experienced before, after the course I was left extremely curious
about what exactly had happened to me during those ten days. For months afterward, I
spent my mornings and evenings wading through the research in the field of
meditation. I was specifically focused on white papers dealing with the effects of a
10-day Vipassana course on the brains of participants. The research in this area
is limited. The quality research which does exist usually uses a sample of highly
experienced meditators rather than beginners and/or self-reports rather than
objective measures.

Over the past year, I have tried to write down what I experienced during that first
10-day course and the 10-day courses I have taken since, corroborating my experience
with research that does exist regarding meditation and the brain.

\textbf{Proviso:} S.N. Goenka, the principal teacher of Vipassana meditation,
actively dissuades students from precisely the sort of brain-centred biological
inquiry presented in this paper.

\begin{quote}
  \textbf{The brain itself is just a physical organ. As you deal with other parts of
    the body, you deal with the brain in the same way, that's all. Nothing special to
    do with the brain. But the mind is totally different. In the West, all importance
    is given to the brain as if the mind is located here. Nothing doing, it is
    everywhere. The mind is in the whole body. So give attention to the whole
    body.} --- S.N. Goenka \cite{goenkabrain}
\end{quote}

This paper does not contradict Mr. Goenka's sentiment, but instead acts as a starting
point for readers who view the function of the brain as central to the activity of
the mind.

\textbf{Disclaimer:} Although my primary field of research is within the field of
biology, my work is far removed from neuroscience. I have tried to simplify the research available so as to better
understand it. If there are any corrections to make or editing in terms of the
content, please feel free to get in touch with me.


\section{Brain Function and Anatomy}

Before dissecting the experience of meditation as it pertains to the brain, this
section describes the different parts of the brain that have shown up in
scientific literature as correlated to meditative practices. Recent neuroscience divides the
brain into the reified geography of the brain, its anatomy, and the abstract concepts
governed by brain activity, its function. Between concrete physiology and abstract
functional outcomes exist networks of cooperative structures which correspond to general
high-level activities of the brain.

The Default Mode Network (DMN) is the constellation of regions which fire when
the brain is not engaged in any external or goal-oriented
tasks. As the name suggests, the brain is in its ``default
mode''. \cite{defaultnetworkadaptive} The DMN is anticorrelated to the Central Executive
Network (CEN), which is active during externally-directed, high-level cognitive
functions. \cite{saliencenetwork} This anticorrelation between the DMN and the CEN is
governed by the Salience Network (SN), the collection of regions in the brain which
help decide which stimuli deserve our attention. The SN acts as a switch between the
internally-directed DMN and the externally-directed CEN. \cite{saliencenetwork}

\todo{insert switching diagram}


\subsection{Default Mode Network}

The DMN is comprised of specific anatomy including portions of the mid-line of the
brain, an evolutionarily primitive area related to memory and emotion, and
structures in the cortex, an evolutionarily recent part of the brain containting the
executive and higher order functions. \cite{defaultnetworkanatomy} These functionally
connected regions are involved in the neurological basis
of the self, considering the mental states of others, remembering the past, and
imagining the future.

\subsubsection{The neurological basis of the self}

The regions involved in an individual's conceptual ``self'' are the Posterior
Cingulate Cortex (PCC), Ventromedial Prefrontal Cortex (vmPFC), and the Inferior
Parietal Lobule (IPL). \cite{defaultnetworkadaptive}

Cerebral blood flow and metabolic rate is 40\%
higher in the PCC compared to the average across the brain, making it one of the most
active regions of the brain. \cite{pccrole}

The PCC has been thought to mediate interactions between emotion and memory. Under
fMRI, it consistently lights up when a person recollects something from their
life. The strength of PCC brain activity in such an experiment varies depending on
the emotional importance of the memory recalled. It is also activated by emotional
stimuli, both positive and negative, and acts as an interface between the external
world and the individual by gauging the importance of different
stimuli. \cite{pccemotion}

\todo{ insert image of ``MRI of DMN'' }
\textit{Magnetic Resonance Image of the areas of the brain in the Default Mode Network.}
http://www.frontiersin.org/Neurotrauma/10.3389/fneur.2013.00016/full

The vmPFC is involved in creating a conceptual self by self-related processing
and the assignment of personal significance to self-related
information. \cite{dmnself} The vmPFC is paired with the Dorsomedial Prefrontal
Cortex (dmPFC), which assists in the creation of the conceptual self through an
individual's consideration of ``others'', to construct the Medial Prefrontal Cortex
(mPFC).

\todo{explain IPL?}

It is hypothesized that the activation of these three regions of the brain (PCC,
vmPFC, and IPL) are responsible for providing a sense of self --- a subjective agent
in space and time. It functions as a network in which phenomena such as self
representations are accessible via the PCC \todo{add PCC reference} and are gated into conscious awareness by
activity in the mPFC, as influenced by changing internal and external demands. The
PCC acts as a brain-wide connectivity hub, through which a unitary notion of self is
created by a large scale integration of DMN activity. \cite{mappingself}

\subsubsection{Thinking about others}

The \textit{Theory of Mind} is a body of work in philosophy and psychology which
describes an individual's ability to consider the mental states of others, such as
appreciating another individual's false beliefs (knowledge based on incorrect or
outdated information). The Theory of Mind captures not only one's ability to
attribute beliefs, desires, and emotions to others but also to oneself --- and to
differentiate the two. \cite{autistictheoryofmind}

In addition to The Theory of Mind, the dmPFC is involved in empathy, moral reasoning,
and altruistic behavior. \cite{defaultnetworkadaptive,dmpfcothers,dmpfcaltruism}

\subsubsection{Autobiographical memory and future simulations}

The DMN is also involved in remembering the past, imagining the future, and story
comprehension. \cite{defaultnetworkadaptive}

\todo{expand on DMN involvement in past/future/stories. how does this work?}

\todo{Summarize DMN? Maybe introduce why it's relevant to meditation?}

\subsection{Central Executive Network}

The CEN is activated when high-level cognitive tasks or external goal-oriented tasks
are being performed. Executive functions are cognitive processes involved in
cognitive control of behavior. The regions of the brain involved in executive
functions are Dorsolateral Prefrontal Cortex (DL-PFC), Orbito Frontal Cortex (OFC),
and the Posterior Parietal Cortex (PPC).

\subsubsection{Dorsolateral Prefrontal Cortex}

The DL-PFC is a part of the Prefrontal Cortex found in primates, including
humans. The DL-PFC is involved in higher cognitive processes including working memory
(holding different pieces of information, manipulating them, and using them for
tasks), selective attention, cognitive flexibility (switching between tasks) and
planning. It also seems to be involved in social cognition and lying. The DL-PFC can
increase dopamine levels in the brain. \cite{dlpfcmemory,dlpfctasks,dlpfclying}

\subsubsection{Orbito Frontal Cortex}

The OFC is an area found in front of both hemispheres of the brain, just above the
eye. It is again part of the Prefrontal Cortex and is thought to be involved in
decision making through emotion and reward. It also receives input from multiple
sensory modalities and in turn activates the Amygdala and the
Hypothalamus. \cite{theprefrontalcortex,ofcprimates,theorbitofrontalcortex}

\todo{add image of Approximate location of the OFC on an MRI}

\subsubsection{Posterior Parietal Cortex}

The PPC receives information from the auditory, visual and somatosensory systems and
in turn activates the Motor Cortex or DL-PFC. \cite{parietallobes}

\todo{why and how does the PCC receive external information and activate the motor cortex/DL-PFC? expand.}

\todo{Summarize CEN? Relevance to meditation?}

\subsection{Salience Network}

\begin{quote}
  \textbf{Salience: The perceptual quality by which an observable thing stands out
    relative to its environment.}
\end{quote}

The SN is an intrinsically connected large-scale network anchored in the Anterior
Insular Cortex (AIC) and Dorsal Anterior Cingulate Cortex (ACC). Both regions have reached a
high degree of specialization in the great apes. It is the collection of the regions
in the brain that help decide which stimuli deserves our attention. It acts as a
switch between the internally-directed DMN and the externally-directed Central
Executive Network. \cite{saliencenetwork}

\subsubsection{Anterior Cingulate Cortex}

The ACC is the front end of the Cingulate Cortex and collars around the Corpus
Callosum, the band connecting the two hemispheres of the brain. It is the connector
between the emotional (Limbic System) and the cognitive (Prefrontal Cortex) part of
the brain. It is involved in functions such as attention allocation, reward
anticipation, decision making, morality, impulse control, emotional awareness and
registering pain. \cite{accstroop,accreward,snmorality,empathypain,acccognitive} It
also appears to play a role in the regulation of Autonomic functions such as blood
pressure and heart rate. \cite{accbloodpressure}

\subsubsection{Anterior Insular Cortex}

The Insula is a part of the Cerebral Cortex located deep within the Sulcus, the
fissure separating the four lobes of the brain. The AIC physically projects itself
into the Amygdala. It is involved in multimodal sensory processing such as
audio-visual integration tasks, interoceptive awareness (so its activity is directly
related to an individual's sense of internal body states), empathy and conscious
awareness. \cite{aicemotion}

It also plays a role in the regulation of autonomic functions such as bodily
sensations (including judgement of the severity of pain), taste, and control of the
immune system. \cite{aicautonomic}

The AIC and ACC together give rise to our interoceptive and conscious
self-awareness. \todo{add reference justifying statement that AIC+ACC create
  conscious self-awareness}

\subsection{The Limbic System}

The limbic system is a complex set of structures that lies on both sides of the
Thalamus, just under the Cerebrum. It includes the Hypothalamus, the Hippocampus, the
Amygdala, and several other nearby areas. It appears to be primarily responsible for
our emotional life, and has a lot to do with the formation of memories. \todo{add
  reference for limbic system}

\section{The Course}

I applied for my first 10-day Vipassana course after I quit my job. From the time I
was accepted into the course until the first day of the course I was very nervous
about what it might entail. I had never done anything like this (meditation) before
and I was quite happy with the idea that meditation, with its spiritual connotations
and religious mumbo-jumbo, was a waste of my time. In my mind, a silent meditation retreat was
for people who had time to waste --- and I was not one of them. Yet friends had convinced me that I
should give Vipassana a try and in the interval between workplaces I had
time to experiment with a course.

I went for my first course in Chennai, my home town. When I arrived, I had to fill
all the application forms all over again (despite applying online), keep all my
luggage in a locker, and go to my room. I had to share the room with one other
person. I felt quite out of my element and I was certain everyone could see in me
that I was the odd one out, the one that was not supposed to be there. Because I felt
so out of place, I was too scared to actually make conversation with anyone before
silence was enforced.

\subsection{Anapana Meditation}

And so it was. With my brain filled with thoughts about myself, about the people
around me, I started this meditation business. For the first 3.5 days you are asked
to focus your attention on the area below your nostrils and above your upper
lip. That's all. Within the first two days, I had made up elaborate stories about my
fellow meditators, all of them superheroes, tirelessly working to save humanity
together. My default mode network was in overdrive.

Focusing our attention below the nostrils starts with observing your breath coming in
and going out. As the teacher mentions in the late evening discourses, the reason for
starting with your breath is because this the only activity of the body which is both
conscious and unconscious --- it acts as a bridge to the unconscious mind and the
involuntary processes of the body.

During the meditation hours I was trying hard. If I was to examine this practice
empirically, I had to give it an honest shot. I focused on the small patch beneath my
nostrils above my upper lip. To begin with, while trying to observe your breath and
your mind wanders, that is the default mode network working. As you realize your mind
has wandered and bring your attention back to your breath, your salience network is
activated. This happens slowly at first, with your mind wandering for many, many
minutes before you realize you are not with your breath. But within the 3.5 days, the
SN learns quite quickly to bring attention back to the breath. After a few days, this
refocusing almost happens by habit, almost automatically.

As I gave in to this activity, tried to focus more intently, tried to sit still for
longer, there were longer periods of awareness on that patch of skin and, with them,
stranger sensations arising and passing within that physical area. By the third day,
it felt like entire ecosystems were writhing and flopping and crashing, all of them
very alive in the area below my nostrils above my upper lip.

There have been studies
that have shown that once you bring your attention back to breath, the neural
structures involved in the control of the Autonomic Nervous System and attention
start firing more actively (in this case, the DL-PFC --- a part of the CEN and the
Hippocampus and ACC --- both part of the SN). There are also global dampening changes
seen in the brain, particularly within the DMN and the Amygdala, which is involved in
the flight-or-fight (stress) response. These changes have been
termed the ``relaxation response'', as they are antagonistic to the stress
response. This relaxation response can be thought of as a gateway to altered states
of mind. \cite{relaxationresponse}

There has also been research in mice, showing a cluster of nerves called the
pre-Bötzinger Complex (preBötC), found in the brain stem of most mammals, which fires
with every breath taken. This breathing pacemaker seems to work not only for regular
conscious and unconscious breathing but for all kinds of breathing --- such as yawns,
sighs and gasps. The preBötc also appears to play a role in calming and
arousal. \cite{prebotcgeneration} It would be interesting to see the activation in
these neuron clusters during the first three days of a Vipassana course, even among
novice meditators.

\subsection{Vipassana Meditation}

After lunch on the fourth day the instructions for Vipassana are given. During
Vipassana, you transfer your attention from the patch below your nostril to the top
of your head. From there, for two hours, the instructions are given to slowly move
your attention throughout the entire body: ``from the top of the head to the tips of
the toes'', then repeat. The two criteria for observation of bodily sensations are
focused attention (making use of the narrow focus practiced during the Anapana
period) and open monitoring (observing the sensations objectively).

The Salience Network (ACC and AIC) and the Central Executive Network (DL-PFC, OFC,
and PPC) which are involved in the control of attention start firing rapidly. Usually
when a person is engaged in an external task, the OFC and PPC are receiving signals
from all the sensory networks (Visual, Audio, Somatosensory, etc.) and, in response,
continuously send signals to the different motor cortices, Amygdala, and Hypothalamus to
respond by performing a task, feeling an emotion, etc.

But Vipassana is an internal task. The Salience Network is activated by focused
attention, which in turn activates the Central Executive Network. Rather than an
external task, as the CEN is accustomed to, the conscious instruction is to observe
bodily sensation --- and do nothing. So as you sit in silence with your eyes closed,
not moving and scanning your body with your attention, you are doing what the CEN has
always done except attempting not to react.

This is hard at first, with the body involuntarily reacting by jerking and writhing
in involuntary response to the act of observation. Initially, your awareness is
itself jerky and it is hard to observe sensations consciously. There is also the
pain. This eclipses all other sensations. In my case, it was difficult to maintain
awareness of any other sensation with the pain that was emanating from my legs and
back. The OFC is involved in assigning emotion to sensation \cite{ofcemotion} and the
ACC plays a vital role in perception of sensation. These two parts of the brain are
usually functionally connected. In the beginning, as I sat trying to observe
sensations but instead was only aware of the pain, the ACC was firing rapidly and the
OFC then assigned these signals to the Limbic System (amygdala and
hippocampus)... which decided to put me in a less than ideal emotional state.

But as your awareness gets smoother and observations get more objective, this
functional connectivity between the ACC and the OFC slowly starts changing. As you
continue to pay attention to the different sensations you attempt to avoid assigning
emotions to them. Without assigning an emotional response, you can observe with a
non-judgmental attitude, particularly toward unpleasant stimuli. The OFC still fires
but the Limbic System is quiesced and, in turn, you react less. It is as if both your
conscious and unconscious awarenesses are slowly aligning until you experience some
occasions when it feels as if there is nothing but pure, detached awareness moving
through your body (or some parts of your body). Senses are not dulled, however. In
addition to a feeling of emotional detachment, the Insula and the Somatosensory
Cortex, which are involved in sensory awareness, were firing \textbf{more} actively,
not less.

(The somatosensory cortex is located in the gyrus and the parietal lobes: The sensory
homunculus which is a part of the somatosensory cortex is a cortical representation
of the body based on the degree of sensory innervation) \todo{Maybe move sensory
  homunculus diagram and explanation to the anatomy section?}

\todo{add sensory homunculus diagram}
Somatosensory and Motor cortices

Perhaps the strangest experience during a Vipassana course is that of shifts in your
sense of self. The ``self'', as a subjective agent in space and time changes, if at
first only very slightly. As you close your eyes and meditate, it may feel as though
the body has shifted in space or rotated in different directions. During these times,
the vmPFC and PCC become less active. As the meditations get deeper the DMN
occasionally quiets completely (perhaps only for a short duration) while the Insula
fires more rapidly, leading to alternative networks of consciousness arising. It is
as if losing the influence of one's own narrative leads to new insights that are
usually kept from consciousness. It has been found that experiencing this, even for a
few seconds, can lead to lasting changes long after the activity of meditation itself
has stopped. \cite{alteredtraits} In the moment, these experiences can be quite
similar to being on certain halucinogenic drugs or even, in some cases, to ecstatic
seizures where the abnormal activity of the Anterior Insular Cortex leads to
heightened self-awareness, feelings of bliss, and a lack of ambiguity. \cite{cortexbliss}

\section{Conclusion}

I believe attending a 10-day Vipassana course did, in fact, change the habit pattern
of my mind --- at least for a while.

Vipassana quiets the DMN, causes certain parts of the SN and CEN to fire more
rapidly. That is, reacting parts of the network and portions of the network which are
aware of bodily sensation, respectively. This causes changes to a person's behavior
(in the short term, at least) and these changes can often be seen as soon as someone
leaves the course on the tenth day.

But repeated practice is essential for any sustained neuroplasticity since rewriting
many years of habit formation requires more than a ten day course. However, the ten
day course is well-designed and will give someone experimenting with meditation the
minimum observational change neccesary to instill an interest, to keep practicing.

There have been studies looking at both the structural changes in the brain and
functional changes in brain activity, especially to the DMN and the SN. But such
studies are usually only done on meditators who have clocked thousands of hours of
practice, which diverges from the purpose of this paper. \todo{add references with
  evidence of structural changes in the brain after meditation}

How is Vipassana different from simply concentrating on a task, as hard as one can?
When concentrating on any task, your DMN is quieted, the SN and CEN are firing, and you
enter into a state of flow. But flow or no flow, the reacting part of the brain is
also working. And it is working hard. The goal of Vipassana is not simply to
concentrate or two enter a state of flow, but to break old reactionary habit patterns
altogether.

As I left the Vipassana Center, I was really intrigued as to what exactly had
occurred over those ten days. What was it that had changed in my brain? I had to
know. I very quickly noticed, however, that as I left the centre (or even earlier, as
soon as talking was allowed), those very changes started slipping away. But with each
hour-long meditation at home, a tiny bit sticks just a little more. And I get closer
to my answer.

\section{Acknowledgements}

Thank you to Steven Deobald for reviews, edits, and corrections.

\begin{thebibliography}{99}

\section*{References}

\bibitem{goenkabrain}
  S.N. Goenka.
  \url{https://www.vridhamma.org/A-store-house-of-answers-by-Shri-S-N-Goenka}
  \textit{Answers by Mr. S. N. Goenka}

\bibitem{defaultnetworkadaptive}
  Andrews-Hanna, Jessica R.
  \textit{The brain’s default network and its adaptive role in internal mentation.}
  The Neuroscientist: A Review Journal Bringing Neurobiology, Neurology and
  Psychiatry. 18 (3): 251–270. doi:10.1177/1073858411403316. ISSN 1089–4098. PMC
  3553600. PMID 21677128. (2012–06–01).

\bibitem{defaultnetworkanatomy}
  Buckner, R. L.; Andrews-Hanna, Jessica R.; Schacter, D. L.
  \textit{The Brain's Default Network: Anatomy, Function, and Relevance to Disease.}
  Annals of the New York Academy of Sciences. 1124 (1): 1–38. CiteSeerX
  10.1.1.689.6903. doi:10.1196/annals.1440.011. PMID 18400922, 2008

\bibitem{saliencenetwork}
  Menon, V; Toga, A.
  \textit{Salience Network.}
  Elsevier. pp. 597–611, 2015
  ISBN: 978–0–12–397316–0.

\bibitem{pccrole}
  Leech R, Sharp DJ
  \textit{The role of the posterior cingulate cortex in cognition and disease.}
  Brain. 137 (Pt 1): 12–32., July 2013.

\bibitem{pccemotion}
  Maddock, Richard J.; Garrett, Amy S.; Buonocore, Michael H.
  \textit{Posterior cingulate cortex activation by emotional words: fMRI evidence from a
    valence decision task.}
  Human Brain Mapping. 18 (1): 30–41. January 2003.

\bibitem{dmnself}
  Andrews-Hanna, Jessica R.; Smallwood, Jonathan; Spreng, R. Nathan.
  \textit{The default network and self-generated thought: component processes,
    dynamic control, and clinical relevance.}
  Annals of the New York Academy of Sciences. 1316(1):
  29–52. doi:10.1111/nyas.12360. ISSN 1749–6632. PMC 4039623. PMID 24502540. 2014-05-01.

\bibitem{mappingself}
  Davey CG, Pujol J, Harrison BJ.
  \textit{Mapping the self in the brain’s default mode network.}
  Neuroimage. 132:390–397., 2016-05-15

\bibitem{autistictheoryofmind}
  Baron-Cohen, Simon; Leslie, Alan M.; Frith, Uta.
  \textit{Does the autistic child have a ``theory of mind''?.}
  Cognition. 21 (1): 37–46. doi:10.1016/0010-0277(85)90022-8. PMID
  2934210. Pdf. October 1985.

\bibitem{dmpfcothers}
  Isoda, M., \& Noritake, A.
  \textit{What makes the dorsomedial frontal cortex active
    during reading the mental states of others?.}
  Neural basis of social learning, social deciding, and other-regarding preferences,
  51. 2015.

\bibitem{dmpfcaltruism}
  Waytz, A., Zaki, J., \& Mitchell, J. P.
  \textit{Response of dorsomedial prefrontal cortex predicts altruistic behavior.}
  The Journal of Neuroscience, 32(22), 7646–7650. 2012.

\bibitem{accstroop}
  Pardo JV, Pardo PJ, Janer KW, Raichle ME.
  \textit{The anterior cingulate cortex mediates processing selection in the Stroop
    attentional conflict paradigm.}
  Proceedings of the National Academy of Sciences of the United States of
  America. 87 (1): 256–9. doi:10.1073/pnas.87.1.256. PMID 2296583. January 1990.

\bibitem{accreward}
  Bush G, Vogt BA, Holmes J, Dale AM, Greve D, Jenike MA, Rosen BR.
  \textit{Dorsal anterior cingulate cortex: a role in reward-based decision making.}
  Proceedings of the National Academy of Sciences of the United States of
  America. 99 (1): 523–8. doi:10.1073/pnas.012470999. PMC 117593. PMID
  11756669. January 2002.

\bibitem{snmorality}
  Sevinc G, Gurvit H, Spreng RN.
  \textit{Salience network engagement with the detection of morally laden
    information.}
  Social Cognitive and Affective Neuroscience. 12 (7):
  1118–1127. doi:10.1093/scan/nsx035. PMID 28338944. July 2017.

\bibitem{empathypain}
  Jackson PL, Brunet E, Meltzoff AN, Decety J.
  \textit{Empathy examined through the neural mechanisms involved in imagining how I
    feel versus how you feel pain.}
  Neuropsychologia. 44 (5): 752–61. doi:10.1016/j.neuropsychologia.2005.07.015. PMID
  16140345. 2006.

\bibitem{acccognitive}
  Bush G, Luu P, Posner MI.
  \textit{Cognitive and emotional influences in anterior cingulate cortex.}
  Trends in Cognitive Sciences. 4 (6):
  215–222. doi:10.1016/S1364–6613(00)01483–2. PMID 10827444. June 2000.

\bibitem{accbloodpressure}
  Gianaros PJ, Derbyshire SW, May JC, Siegle GJ, Gamalo MA, Jennings JR.
  \textit{Anterior cingulate activity correlates with blood pressure during
    stress.}
  Psychophysiology. 42(6):627–35. 2005.

\bibitem{aicemotion}
  Xiaosi Gu, Patrick R. Hof, Karl J. Friston, Jin Fan.
  \textit{Anterior Insular Cortex and Emotional Awareness.}
  J Comp Neurol. 521(15): 3371–3388. 2013-08-15.

\bibitem{aicautonomic}
  Rolls ET.
  \textit{Functions of the anterior insula in taste, autonomic, and related
    functions.}
  Brain Cogn. 110:4–19. December 2016.

\bibitem{dlpfcmemory}
  Barbey AK, Koenigs M, Grafman J.
  \textit{Dorsolateral prefrontal contributions to human working memory.}
  Cortex. 49 (5): 1195–1205. May 2013.

\bibitem{dlpfctasks}
  Monsell S.
  \textit{Task switching.}
  Trends in Cognitive Sciences. 7 (3):
  134–140. doi:10.1016/S1364–6613(03)00028–7. PMID 12639695. 2003.

\bibitem{dlpfclying}
  Ito, Ayahito; Abe, Nobuhito; Fujii, Toshikatsu; Hayashi, Akiko; Ueno, Aya;
  Mugikura, Shunji; Takahashi, Shoki; Mori, Etsuro.
  \textit{The contribution of the dorsolateral prefrontal cortex to the preparation
    for deception and truth-telling.}
  Brain Research. 1464: 43–52. doi:10.1016/j.brainres.2012.05.004. 2012.

\bibitem{theprefrontalcortex}
  Fuster, J.M.
  \textit{The Prefrontal Cortex}
  Raven Press, New York, 1997.

\bibitem{ofcprimates}
  Rolls, ET.
  \textit{Convergence of sensory systems in the orbitofrontal
    cortex in primates and brain design for emotion.}
  The Anatomical Record Part A: Discoveries in Molecular, Cellular, and Evolutionary
  Biology. 281 (1):1212–25. doi:10.1002/ar.a.20126. November 2004.

\bibitem{theorbitofrontalcortex}
  Price, Joseph L.
  \textit{Chapter 3: Connections of the orbital cortex.}
  Rauch, Scott L.; Zald, David H.
  The Orbitofrontal Cortex. p. 45.
  Oxford University Press, New York. 2006.

\bibitem{parietallobes}
  Martin, R. E.
  \textit{Let’s Get to Know the Parietal Lobes!}
  \url{http://gablab.mit.edu/downloads/Parietal_Primer.pdf}
  \todo{parietal lobes article returns a 404. fix.}

\bibitem{relaxationresponse}
  Lazar SW; Bush George; Gollub RL.; Fricchione, GL.; Khalsa G; Benson H.
  \textit{Functional brain mapping of the relaxation response and meditation.}
  Neuroreport. 11(7):1581–1585. 2000-05-15.

\bibitem{prebotcgeneration}
  J. Muñoz-Ortiz, E. Muñoz-Ortiz, L. López-Meraz, L. Beltran-Parrazal,
  C. Morgado-Valle.
  \textit{The pre-Bötzinger complex: Generation and modulation of respiratory rhythm}
  Neurología (English Edition), ISSN 2173–5808,
  https://doi.org/10.1016/j.nrleng.2018.05.006. 2018.

\bibitem{ofcemotion}
  Rempel-Clower, N. L.
  \textit{Role of Orbitofrontal Cortex Connections in Emotion.}
  Annals of the New York Academy of Sciences,
  1121:72–86. doi:10.1196/annals.1401.026. 2007.

\bibitem{alteredtraits}
  Goleman, Daniel; Davidson, Richard J.
  \textit{Altered Traits}
  ISBN: 9780399184390. September 2018.

\bibitem{cortexbliss}
  Picard, Fabienne.
  \textit{State of belief, subjective certainty and bliss as a product of cortical
    dysfunction}
  Cortex, Volume 49, Issue 9, Pages 2494–2500ISSN 0010 9452,
  https://doi.org/10.1016/j.cortex.2013.01.006. 2013.

\end{thebibliography}

% \appendix*
% \input{sections/appendix1.tex}

\end{document}
