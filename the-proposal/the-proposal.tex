\documentclass[a4paper, amsfonts, amssymb, amsmath, reprint, showkeys, nofootinbib, twoside]{revtex4-1}

\usepackage[english]{babel}
\usepackage[utf8]{inputenc}
\usepackage[colorinlistoftodos, color=green!40, prependcaption]{todonotes}
\usepackage{graphicx}
\usepackage{subcaption}
\usepackage{float}
\usepackage[bottom]{footmisc}
\usepackage{enumitem}
\usepackage{hyperref}

\bibliographystyle{apsrev4-1}
\setlist{noitemsep}

\begin{document}

\title{%
  \large{Vipassana for Hackers} \\
  \Huge{The Proposal} \\
  \large\textit{Version 0.1}
}
\author{Steven Deobald}
\email[Correspondence email address: ]{steven@deobald.ca}
\affiliation{www.vipassana-for-hackers.org}
\date{\today}

\begin{abstract}
Vipassana meditation (as taught by S.N. Goenka in the tradition of Sayagyi U Ba Khin)
is unlike other meditation techniques insofar as its claims to completeness and
outcomes are concerned. Vipassana claims to ultimately explore the entire field of
mind and matter, with the goals of total liberation and full enlightenment. Implicit
within these claims is a complete understanding of human consciousness. These are
difficult claims to prove because the time commitment required to research the
technique orders on multiple decades --- if not multiple generations. This is complicated by the
fact that the time commitment is demanded of both the researcher \textit{and the
subject}. Until now, research on meditation of all kinds has covered only one of
two fields: (1) controlled experiments which must inherently rely on superficial data
gathered from beginners and (2) observational exploratory research of monks --- expert
meditators who have dedicated their lives to the practice. I propose bridging this
gap by submitting myself to the middle ground. While remaining a layperson, I will
commit to a high ratio of waking medition hours for the rest of my life. Individually
and internally, I will conduct qualitative research into the consequences of
Vipassana meditation and the nature of consciousness while collectively and
externally pledging myself as a subject for long-term quantitative
studies with a broader community of researchers.

\todo{add references to (1) and (2)}

\end{abstract}

\keywords{neuroscience, psychology, vipassana, meditation}

\maketitle

\listoftodos

\section{Introduction}

Research into the effects of meditation has been conducted in earnest for half a
century but the quality of research in this field varies wildly. Randomized
controlled trials were missing from much research conducted during the first few
decades of meditation study. The importance of active controls was often missed even
when randomized controlled trials were attempted. Double-blind studies are inherently
impossible with meditation research; a subject will always know if she is receiving
meditation instruction or an active control instruction. \cite{alteredtraits}

Add to these difficulties the very nature of meditation research itself. There are
many techniques of meditation and it is very important to capture the specific
technique under study to make meaningful assertions about its
effects. \cite{alteredtraits} However, even within the definition of a single
meditation technique there exist variations in instruction between
teachers and each student's comprehension of the instructions
received. \cite{davidson2015} Even if researchers could cement (or at least
accurately record) semantics and terminology, the subject of study is often unclear:
Are we looking for health benefits? Increases in productivity? Increases in
intelligence? How long these effects persist? How much do we want to explore concrete
hypotheses versus exploratory analysis of long-term effects? How much can be learned
about the nature of consciousness? Can these learnings be measured objectively or
even communicated meaningfully?

Because Vipassana is globally standardized, it affords researchers with definitive
solutions to the difficulties presented by varying teaching methodologies. In
exchange for this, the difficulty of long-term study is compounded by the intrinsic
seriousness demanded of a Vipassana practitioner: the concept of ``continuous
piercing awareness of constantly changing bodily sensation'' is not only a
requirement of serious Vipassana practice but could very well act as a surrogate
description of the practice itself. This seriousness poses obvious difficulties and
it is the intention of my study to begin breaking ground in solutions to those
difficulties.

This study will prove significant in three fields of research:

First, and most accessible, is the continued quantitative research of meditation in
the broader neuroscience and psychology disciplines, where my participation will be
more as subject than researcher.

Second is the qualitative research into the long-term consequences of
Vipassana meditation, what it reveals about the nature of human consciousness, and
reusable techniques for its exploration --- a field currently consisting of a bridge
between psychology, philosophy, linguistics, and contemplative studies.

Third is the exploratory research intended to objectively define
consciousness. As such research must pertain to all forms of consciousness it
therefore includes all non-human forms of consciousness. Findings will inform our
understanding of the Mind-Body Problem of psychology and philosophy, the entire field
of bioethics, and --- perhaps most importantly --- the nascent field of Artificial
Consciousness as a subfield of Artificial Intelligence. \cite{hildt2019} As of this
writing, this overarching integral field of study has yet to emerge and has no name
as a discipline.

\section{Problem Statement}

A large scale controlled study of beginner-to-intermediate
students of Vipassana, each practicing a minimum of two hours daily, is
possible. Randomization will not be possible in such a study as the students
self-select this

- middle ground: me

(vs. monks)

\subsection{Overview (?)}
\subsection{Hypothesis}

\section{Objectives and Aims}

\subsection{Overall Objective}
\subsection{Specific Aims}

\section{Background and Significance}

\subsection{Preliminary Research Review}
- summary of related research
- strengths and weaknesses
- justification
--- what hasn't been done by others?
--- why is this research necessary?

\subsection{Why Vipassana?}

1. ``complete'', standardized, global, multilingual, 100\% free
2. mundane (sleep) vs. supramundane (total eradication of suffering)



3. who is steven deobald

\subsection{Why now?}

\section{Research Design and Methods}
\subsection{OVERVIEW}

\subsection{Study Design / Research Method}
- case study type?

\subsection{POPULATION AND STUDY SAMPLE}
\subsection{SAMPLE SIZE AND SELECTION OF SAMPLE}
\subsection{SOURCES OF DATA}
\subsection{COLLECTION OF DATA}
\subsection{EXPOSURE ASSESSMENT}
\subsection{DATA MANAGEMENT}
\subsection{DATA ANALYSIS STRATEGIES}

\subsection{Participants}
- people & roles

\subsection{ETHICS AND HUMAN SUBJECTS ISSUES}
\subsection{TIMEFRAMES}

\section{STRENGTHS AND WEAKNESSES OF THE STUDY}
\section{PUBLIC HEALTH SIGNIFICANCE}
\section{BUDGET AND MOTIVATION}
- resources required

\section{Conflict of Interest Statement?}

\section{REFERENCES}
\section{APPENDICES}
\subsection{Appendix 1: Proposed Daily Schedule}

\section*{Acknowledgements}

Thank you to Preethi Govindarajan for reviews, edits, and corrections.


\section*{References}

\begin{thebibliography}{99}

\bibitem{alteredtraits}
  Goleman D., Davidson R.J.
  \textit{Altered Traits.}
  ISBN: 9780399184390. September 2018.

\bibitem{davidson2015}
  Davidson R.J., Kaszniak A.W.
  \textit{Conceptual and methodological issues in research on mindfulness and meditation.}
  American Psychologist, Volume 70(7), Pages 581-92. October 2015.
  \url{https://doi.org/10.1037/a0039512}

\bibitem{hildt2019}
  Artificial Intelligence: Does Consciousness Matter?
  \todo{stop doing this and use bibtex}

\bibitem{entropicrevisited}
  Carhart-Harris, Robin L.
  \textit{The entropic brain - revisited}
  NEUROPHARMACOLOGY, Vol: 142, Pages: 167-178, ISSN: 0028-3908, 2018.


\end{thebibliography}

% \appendix*
% \input{sections/appendix1.tex}

\end{document}
