\documentclass[a4paper, amsfonts, amssymb, amsmath, reprint, showkeys, nofootinbib, twoside]{revtex4-1}

\usepackage[english]{babel}
\usepackage[utf8]{inputenc}
\usepackage[colorinlistoftodos, color=green!40, prependcaption]{todonotes}
\usepackage{graphicx}
\usepackage{subcaption}
\usepackage{float}
\usepackage[bottom]{footmisc}
\usepackage{enumitem}
\usepackage{hyperref}

\bibliographystyle{apsrev4-1}
\setlist{noitemsep}

\begin{document}

\title{%
  \large{Vipassana for Hackers} \\
  \Huge{The Proposal} \\
  \large\textit{Version 0.1}
}
\author{Steven Deobald}
\email[Correspondence email address: ]{steven@deobald.ca}
\affiliation{www.vipassana-for-hackers.org}
\date{\today}

\begin{abstract}
Vipassana meditation (as taught by S.N. Goenka in the tradition of Sayagyi U Ba Khin)
is unlike other meditation techniques insofar as its claims to completeness and
outcomes are concerned. Vipassana claims to ultimately explore the entire field of
mind and matter, with the goals of total liberation and full enlightenment. Implicit
within these claims is a complete understanding of human consciousness. These are
difficult claims to prove because the time commitment required to research the
technique orders on multiple decades --- if not multiple generations. This is complicated by the
fact that the time commitment is demanded of both the researcher \textit{and the
subject}. Until now, research on meditation of all kinds has covered only one of
two fields: (1) controlled experiments which must inherently rely on superficial data
gathered from beginners and (2) observational exploratory research of monks --- expert
meditators who have dedicated their lives to the practice. I propose bridging this
gap by submitting myself to the middle ground. While remaining a layperson, I will
commit to a high ratio of waking medition hours for the rest of my life. Internally,
I will conduct qualitative research into the nature of consciousness while
simultaneously exposing myself as a subject to long-term quantitative studies.

\todo{add references to (1) and (2)}

\end{abstract}

\keywords{neuroscience, vipassana, meditation}

\maketitle

% \listoftodos

\section{Why Vipassana?}

zig zig zig



\section*{Acknowledgements}

Thank you to Steven Deobald for reviews, edits, and corrections. \texttt{Brain\_human\_normal\_inferior\_view.svg} and
\texttt{Brain\_human\_lateral\_view.svg} by Patrick J. Lynch, medical illustrator, CC-BY 2.5 \cite{brainsvg}


\section*{References}

\begin{thebibliography}{99}

\bibitem{entropicrevisited}
  Carhart-Harris, Robin L.
  \textit{The entropic brain - revisited}
  NEUROPHARMACOLOGY, Vol: 142, Pages: 167-178, ISSN: 0028-3908, 2018.


\end{thebibliography}

% \appendix*
% \input{sections/appendix1.tex}

\end{document}
