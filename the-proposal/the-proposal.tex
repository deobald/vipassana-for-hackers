\documentclass[a4paper, amsfonts, amssymb, amsmath, reprint, showkeys, nofootinbib, twoside]{revtex4-1}

\usepackage[english]{babel}
\usepackage[utf8]{inputenc}
\usepackage[colorinlistoftodos, color=green!40, prependcaption]{todonotes}
\usepackage{graphicx}
\usepackage{subcaption}
\usepackage{float}
\usepackage[bottom]{footmisc}
\usepackage{enumitem}
\usepackage{hyperref}

\setlist{noitemsep}

\begin{document}

\title{%
  \large{Vipassana for Hackers} \\
  \Huge{The Proposal} \\
  \large\textit{Version 0.1}
}
\author{Steven Deobald}
\email[Correspondence email address: ]{steven@deobald.ca}
\affiliation{www.vipassana-for-hackers.org}
\date{\today}

\begin{abstract}
Vipassana meditation claims to ultimately explore the entire field of
mind and matter, with the goal of total liberation from suffering. Implicit
within this claim is a complete understanding of human consciousness.
This is a difficult claim to prove or disprove because the time commitment
required to research the technique orders on multiple decades
--- if not multiple generations.
Only a longitudinal study has any meaning.
This is complicated by the fact that the time commitment is demanded of
both the researcher \textit{and the subject}.
Until now, research on meditation of all kinds has covered only one of two fields:
(1) controlled experiments which must inherently rely on superficial data
gathered from beginners \cite{https://doi.org/10.1002/jclp.20761,
  https://doi.org/10.1037/a0018441,
  https://doi.org/10.1093/scan/nss054,
  https://doi.org/10.31231/osf.io/hzv65}
and (2) observational exploratory research of monks --- expert
meditators who have dedicated their lives to the practice.
\cite{https://doi.org/10.1073/pnas.0407401101,
https://doi.org/10.1371/journal.pone.0073417}

I propose bridging this
gap by submitting myself to the middle ground. While remaining a layperson, I will
commit to a high ratio of waking medition hours.
Individually and internally, I will conduct qualitative research into the consequences of
Vipassana meditation and the nature of consciousness.
Collectively and externally I will pledge myself as a subject for long-term quantitative
studies with a broader community of researchers.

\end{abstract}

\keywords{neuroscience, psychology, vipassana, meditation}

\maketitle

\listoftodos

\section{Introduction}

Research into the effects of meditation has been conducted in earnest for half a
century but the quality of research in this field varies wildly. Randomized
controlled trials were missing from much research conducted during the first few
decades of meditation study. The importance of active controls was often missed even
when randomized controlled trials were attempted. Double-blind studies are inherently
impossible with meditation research; a subject will always know if she is receiving
meditation instruction or an active control instruction. \cite{goleman2017altered}

Add to these difficulties the very nature of meditation research itself. There are
many techniques of meditation and it is very important to capture the specific
technique under study to make meaningful assertions about its
effects. \cite{goleman2017altered} However, even within the definition of a single
meditation technique there exist variations in instruction between
teachers and each student's comprehension of the instructions
received. \cite{davidson2015conceptual} Even if researchers could cement (or at least
accurately record) semantics and terminology, the subject of study is often unclear:
Are we looking for health benefits? Increases in productivity? Increases in
intelligence? How long these effects persist? How much do we want to explore concrete
hypotheses versus exploratory analysis of long-term effects? How much can be learned
about the nature of consciousness? Can these learnings be measured objectively or
even communicated meaningfully?

Because Vipassana \footnote{For the remainder of this research proposal,
  ``Vipassana'' will always refer to the specific technique ``Vipassana as taught by S.N. Goenka in the
  tradition of Sayagyi U Ba Khin'' and lineage-identical instruction, such as that
  taught by Ledi Sayadaw (in writing) or other students of U Ba Khin contemporary
  to S.N. Goenka, unless noted otherwise.} is globally standardized, it affords
researchers with definitive
solutions to the difficulties presented by varying teaching methodologies. In
exchange for this, the difficulty of long-term study is compounded by the intrinsic
seriousness demanded of a Vipassana practitioner: the Pāli concept of \textit{ātāpī
  sampajāno satimā} (continuous piercing awareness of constantly
changing bodily sensation \cite{goenka1999discourses, analayo2003satipatthana})
is not only a
requirement of serious Vipassana practice but could very well act as a surrogate
description of the practice itself. This seriousness poses obvious difficulties and
it is the intention of my study to begin breaking ground in solutions to those
difficulties.

This study will prove significant in three fields of research:

First, and most accessible, is the continued quantitative research of meditation in
the broader neuroscience and psychology disciplines, where my participation will be
more as subject than researcher.

Second is the qualitative research into the long-term consequences of
Vipassana meditation, what it reveals about the nature of human consciousness, and
reusable techniques for its exploration --- a field currently consisting of a bridge
between psychology, philosophy, linguistics, and contemplative studies.

Third is the exploratory research intended to objectively define
consciousness. As such research must pertain to all forms of consciousness it
therefore includes all non-human forms of consciousness. Findings will inform our
understanding of the Mind-Body Problem of psychology and philosophy, the entire field
of bioethics, and --- perhaps most importantly --- the nascent field of Artificial
Consciousness as a subfield of Artificial Intelligence. \cite{hildt2019artificial} As of this
writing, this overarching integral field of study has yet to emerge and has no name
as a discipline.

\section{Problem Statement}

\subsection{Overview}

There is a dearth of high-quality scientific meditation research.
Due to decades of misunderstanding the subject, it's estimated that
--- of the 50,000+ papers published on the broad topic of ``meditation'' ---
only a few dozen conform to scientific rigour.
\cite{goleman2017altered,semanticscholar2024meditation}

There are exceptions. Notably, Richard Davidson's 2002 study on Minyur Rinpoche and
the subsequent EEG and fMRI studies of serious Tibetan monks.
\cite{https://doi.org/10.1073/pnas.0407401101}
There have also been effective double-blind trials of householders (the laity),
predominantly studying Mindfulness-Based Stress Reduction (MBSR) and
Mindfulness-based Cognitive Therapy (MBCT), which represent two very mild
techniques of meditation.

Sampling from the extremes is a natural consequence of the accessibility of subjects.
Contemplative monks, serious in their meditation practice, are not easily accessed.
Modern householders tend to peak at 2 or 3 hours of meditation per day,
due to other responsibilities.
This is the dichotomic choice in meditation research:
either sporadically studies monks, who can have tremendous meditative capacity
--- or regularly studies householders, who have little to none.

Bridging this monastic / householder divide requires individuals willing to
(a) dedicate themselves to long periods (years or decades) of continuous meditation
practice with a single technique and
(b) submit themselves to continuous scientific inquiry.

\subsection{Hypothesis}

As far as quantitative measures, such as future EEG, MEG, and fMRI readings, are concerned,
this study hypothesizes that readings taken from subjects will progress toward readings
found in ``expert'' groups of previous studies.
\cite{https://doi.org/10.1073/pnas.0407401101,https://doi.org/10.1371/journal.pone.0073417}
See \textit{RESEARCH DESIGN AND METHODS, Expertise}.

However, both quantitative and qualitative measures will be used primarily for
initial discovery and exploration.
Since no other study of this kind has ever been run, it is not possible to meaningfully
predict outputs.
See \textit{RESEARCH DESIGN AND METHODS, Methodological Issues}.

\section{Objectives and Aims}

\subsection{Overall Objective}

As an ``ultralong'' longitudinal study of continuous practice,
the primary goal is open discovery.

Most pre-existing studies on meditation focus on narrow outputs of the
practitioners' meditation activity.
Qualitiative examples include effects on emotional regulation, negative self-belief,
depression, anxiety, and amygdala response outside of meditation.
\cite{https://doi.org/10.1093/scan/nss054,
  https://doi.org/10.1037/a0018441,
  https://doi.org/10.31231/osf.io/hzv65}
Quantiative examples include discovery of high-amplitude gamma synchrony during meditation,
higher Parietal-Occipital EEG gamma activity during NREM sleep,
Heart Rate (HR) and High-Frequency Heart Rate Variability (HF-HRV) as useful proxies for effort,
and meditation experience correlated to reduced Default Mode Network (DMN) activity.
\cite{https://doi.org/10.1073/pnas.0407401101,
  https://doi.org/10.1371/journal.pone.0073417,
  https://doi.org/10.1016/j.ijpsycho.2015.04.017,
  https://doi.org/10.1073/pnas.1112029108}

This ultralong study will, instead, examine all aspects of prolonged practice of
Vipassana meditation.
Qualitative outputs will begin after the first quarter of the study.
Quantitative outputs will emerge as long-term access to research equipment becomes available.

\subsection{Specific Aims}

The study will produce qualitative writing describing the research-subject's experience.

The study will produce quantitative measures, based on equipment availability.

The study will also act as a template study methodology for others to follow,
or for other researcher-subjects to join if the study proves successful in
uncovering new discoveries.

There has been limited, unprofessional writing on the topic of the dangers of meditation.
\cite{ingram2020mastering,https://doi.org/10.1080/03060497.1983.11084578}
Even though these writings focus on Christian ideals, struggles of faith,
and meditative concepts (such as ``The Dark Night of The Soul''),
which have no relationship to meditation whatsoever,
they do arise in discussions of meditation from time to time.
As a result, it is a specific aim of this study to determine if there are
psychological or physical dangers presented by prolonged meditation practice.

\section{Background and Significance}

\subsection{Preliminary Research Review}

Related research, as mentioned previously, falls on the spectrum of exploratory
research conducted on ``expert'' (50,000 hours of practice or more) meditators
in various traditions or acute, quantitative research conducted on novice
(0 - 10,000 hours) or intermediate (10,000 - 50,000 hours) meditators.

The strength of each form of study is demonstrated in its results.
Exploratory research has given us a clear understanding that the brains of
very experienced meditators function in a manner very different from the
societal mean.
Quantitative research with a narrow hypothesis has provided repeated
statistical evidence that meditation practices do change the minds and bodies
of the participant, and that those changes are measurable.
Statistical research has also taught us that the effects of many
meditation practices become more pronounced with a prolonged practice period.

Longitudinal studies have largely not been conducted in the various fields
of meditation research.
Where they have been conducted, ``longitudinal'' may only represent a period
of nine months.
Generally, these longitudinal studies focus on larger groups of novice meditators.
While useful in its own right, this is equivalent to studying hobby pianists
for long periods of time: none of the subjects will ever master the piano.

True longitudinal study of an individual (or individuals) will allow for a
new form of exploratory research, which will itself uncover new questions
we might ask of meditation.
We can then probe with further statistical research once we know what
questions we want to ask of the technique or techniques under study.

\subsection{Why Vipassana?}

Vipassana is a ``complete'' system of meditation, based on two premises:

1. that every mental and emotional formation in the consciousness of an individual
corresponds to an equivalent physical formation in the bodily sensations
of that individual and

2. that the totality of physical, bodily sensations are open to exploration by
the method of self-directed human awareness

These premises can be experientially validated, and are thus known to be true.
As with all experience-oriented factual information found in the fields of
consciousness studies, we must operate based on the assumption that if a
third party wishes to validate these premises, they can and will do so
experientially.
This presents a difficulty, but not an insurmountable one, as we'll see below.

\subsubsection{Standardization and Access}

Vipassana meditation is a globally standardized meditation practice.
It is taught based on the same instructions across the globe,
and has been translated into dozens of languages, covering the vast
majority of global society.
Those instructions are taught verbally, via audio and video recording,
and do not require the participant to be literate in the language they speak.
Introductory courses in Vipassana meditation are all identical, covering
a 10-day period in which meditators participate in complete silence,
devoid of external input and confounding variables.

These introductory courses (as with all Vipassana meditation courses)
are taught completely free of charge, making access universal.

The only limitation to access is the mental and physical fitness of
individuals applying to take Vipassana meditation courses.
Course management will, at their discretion, request someone not to
take a course if they suffer from mental illness, for example.

\subsubsection{Purpose of Practice}

The intended purpose of Vipassana meditation is inherently supramundane:
``the total eradication of suffering.''
While this purpose may appear unreasonable or unattainable to skeptics,
it is not scientific to judge the potential outcomes of such a meditation
based on our assumptions.
We must instead test the methodology of Vipassana meditation to see
how it holds up to these claims, and to what degree.

Conversely, the purpose of many contemporary meditation techniques
(particularly those studied in existing literature) is inherently mundane:
to improve sleep, to reduce anxiety, to reduce PTSD, and so on.
While some of these effects may be visible in the study of Vipassana
meditation, they are not the end goal of the technique under study.

\subsection{Why now?}

Instruction in Vipassana meditation has been available outside of
Burma / Myanmar since 1969.
However, it has taken five decades of proliferation at the hands of
volunteers to make Vipassana meditation accessible to large portions of
the globe.
Even though there are permanent Vipassana Centres (teaching schools) in
many countries worldwide, there is still extremely limited presence in
Africa and the Middle East, though there is widespread interest there.

10-Day introductory courses in Vipassana meditation have long waitlists
in all of these Centres.
Demand for instruction in Vipassana meditation far outstrips supply.

Vipassana instruction is clearly on a path of growth, at present.
With a globally-accessible technique of meditation which is increasingly
discussed within the scope of popular culture, it is an appropriate
time to study its effects for the benefit of the global population.
\cite{marriage2024theretreat,muraskin2024perils,sacchet2024advanced}

Outside the scope of Vipassana meditation, the broader topic of
``meditation'' has received a great deal of confused analysis in
recent years.
Many authors will make the mistake of confusing their own subjective
analysis of their personal experiences as somehow objective, or
present it as such.
\cite{ingram2020mastering,https://doi.org/10.1007/s12671-020-01389-4}
It is also an appropriate time to clarify the conclusions of such
confused analysis.

\section{Research Design and Methods}

\subsection{Overview}

As the structure of studies conducted on Vipassana is inherently very difficult, due to
the strict nature of Vipassana meditation instruction, we must first examine which kinds of
studies are possible, which are not, and what will make possible studies worthwhile.

\subsubsection{Expertise}

A large scale controlled study of beginner-to-intermediate
students of Vipassana, each practicing a minimum of two hours daily, is
possible. Randomization will not be possible in such a study as the students
self-select this meditation technique for themselves. The recommendation to practice
the technique ``a minimum of two hours daily'' is a component of instruction, given
to students even on the most basic 10-day introductory
course. \cite{dhammaorg2017guidelines,goenka2001importance} As such, it is possible
to conduct such a study on students who have dedicated themselves to the practice of
Vipassana in the manner recommended --- and an increasingly large number of Vipassana
meditators do so, worldwide.

However, such a study suffers from the very nature of
the instruction and its target audience. On the lifelong scale of Vipassana
practice, the spectrum spans from a minimum of laypeople practicing two hours per day to
a maximum exemplified by renunciates (monks and nuns) who practice up to twenty-four
hours per day once they
reach the advanced stage where deep sleep no longer occurs. It is
also worth noting that a junior renunciate will still have less \textit{experience}
than a senior lay meditator, which means that the only objective measure of
experience is total number of hours meditated. \cite{goleman2017altered}

Meditation research often has difficulty defining and capturing \textit{expertise},
however, and total hours of experience is still confounded by the rate in which those
hours of experience are accumulated (hours per day). We must capture both, as accurately as
possible. Renunciates potentially have both a high hours-per-day rate and extended
(lifelong) duration of experience. The existence of renunciates forces the very
broad category of \textit{expert} meditators into the terrority of 100,000 hours
of practice, even if we limit practice hours to formal, sitting meditation and
estimate twelve (12) hours per day. If we take the much more modest rate of six (6)
hours per day, assume an adult renunciate can learn Vipassana, at the earliest, from
approximately 20 years of
age, and is now of an age when they may participate in an extended study (say, 40-60
years of age), we arrive at roughly 60,000 hours of practice.
We can reduce this number to 50,000 to match the upper-bound of groups
categorized in studies by Lumma and Brewer.
\cite{https://doi.org/10.1016/j.ijpsycho.2015.04.017,https://doi.org/10.1073/pnas.1112029108}
If we use the oft-cited
``10,000 hours'' measurement for competence in a subject, we might reasonably
describe our rough spectrum of expertise as such:

\begin{itemize}
	\item Beginner: 0-10,000 hours
	\item Intermediate: 10,000-50,000 hours
	\item Expert: 50,000+ hours
\end{itemize}

As a consequence, even studies which claim to observe ``expert'' or ``long-term''
Vipassana meditators are often predominantly observing beginners (7.9-8.6 mean years
of experience with 2 hours of daily practice). \cite{chiesa2010vipassana}

\subsubsection{Methodological Issues}

This is in addition to a large number of other
methodological issues with studies of Vipassana, as described by Alberto Chiesa in
\textit{Vipassana Meditation: Systematic Review of Current Evidence},
2010. \cite{chiesa2010vipassana} These include the lack of: study replication,
randomized trials, active plus inactive controls, and double/single blinding.

Chiesa also notes that it would be beneficial to capture data
``both from a clinical and from a neuro-imaging point of view'',
including functional magnetic resonance imaging (fMRI)
and electroencephalography (EEG), for both short-term outcomes
(altered states) and long-term outcomes (altered traits \cite{davidson1977role}) to
improve future Vipassana research. \cite{chiesa2010vipassana}
Optically-pumped magnetometer (OPM)-based magnetoencephalography (MEG) performed using a
mobile helmet capable of operating at room temperature \cite{boto2017new},
in development since Chiesa's
paper was published, may permit it as a third neuro-imaging technique. Even older MEG
technology based on superconducting quantum interference devices (SQUIDs) could
potentially be used. The restriction of SQUID-based MEG is that subjects must remain
extremely still, but this is all but a requirement of serious Vipassana meditation
anyway.

Because Vipassana does not permit any form of imagination, any attempt to analyze it
through philosophical phenomenology becomes a hindrance to actually practicing
Vipassana. \cite{patrik1994phenomenological} The two are mututally exclusive, as is
any attempt to contemplate Vipassana, the phenomenology of the mind, or the
technique, during the actual practice of Vipassana. I have previously discussed this
apparent paradox in Appendix A of
\textit{Vipassana for Hackers, Paper One: Curious Mechanics}.
\cite{deobald2017curious}

Last, statistical study of Vipassana, at the intermediate level, is currently not
possible. Meditators who are in the process of transitioning from 10,000 hours of
experience to 50,000 hours of experience are not readily available as research
subjects in significant numbers. Because we have not studied such meditators, we are
as of yet not exactly sure what we might be studying when we do study
them. V.S. Ramachandran put this most succinctly:

\begin{quotation}
  ``I can't think of a single discovery of disease which had more than one initial
  sample. ... You can't do statistical analysis of an initial discovery.''
  \cite{ramachandran2019relevance}
\end{quotation}

\subsubsection{Meditation as Research Tool}

A lifelong study of a single, increasingly-experienced Vipassana meditator
is, in essence, an exploratory process of discovering what researchers might be
bothered to study in a controlled trial with both active and inactive controls and
(single) blinding, over a much larger sample. This exploratory process has been
described by Goleman and Davidson in \textit{Altered Traits}:

\begin{quotation}
  ``Perhaps one day an ultralong study will give us the equivalent of video on how
  altered traits emerge. For now, as the Brewer group conjectured, meditation seems
  to transform the resting state---the brain's default mode---to resemble the
  meditative state.

  Or, as we put it long ago, the after is the before for the next
  during.'' \cite{goleman2017altered}
\end{quotation}

The demands of such an ``ultralong study'' were echoed by Harari in \textit{21
  Lessons for the 21st Century}:

\begin{quotation}
  ``Some universities and laboratories have indeed begun using meditation as a
  research tool rather than as a mere object for brain studies. Yet this process is
  still in its infancy, partly because it requires an extraordinary investment on the
  part of the researchers.'' \cite{harari201821}
\end{quotation}

The line between ``mere object for brain studies'' and ``research tool'' begins to
blur when we consider using meditation for both simultaneously. However, we would be
remiss not to
gather whatever objective EEG/fMRI/MEG data from the long-term subject-researcher
when that data is so readily available.

\subsubsection{Summary}

Because Vipassana meditators are inherently self-selecting, double blind trials are not
possible, nor are randomized trials. Single blinding is possible, and should be
employed in future studies where it is feasible. Combination active/inactive controls
are also possible for all \textit{statistical} studies on Vipassana.

The study proposed is an ultralong (lifelong) study of an individual --- myself.
There is potential for others to join this ultralong study, should it show promise.
As far as quantitative data gathered from brain scans is concerned, I am effectively
making myself available as a human guinea pig. As far as qualitative data is
concerned, research will take the form of a deep case study. As I will represent
Ramachandran's ``initial sample'', statistical analysis will not be relevant.
If the study proves productive, future studies targeting a specific hypothesis about
Vipassana in a large sample can use learnings from this study in their statistical
structure.

\subsection{Study Design}

\subsubsection{Qualitative Measures}

Qualitative research into the specific nature of human consciousness or the broader
nature of consciousness as it applies to any organism or artificial intelligence can
only be performed directly. In this case, a case study performed by an individual
(myself) will explore the question of consciousness and the mind-body problem
directly, through Vipassana meditation, for a prolonged period of time.
Many theories and questions currently exist regarding
consciousness, such as \textit{Theory of Mind}, \textit{Theory of Panpsychism},
\textit{Theory of the Entropic Brain},
the \textit{Simulation Hypothesis}, etc. This study will not address any one of the
many theories of consciousness
directly, nor will it absorb any specific hypothesis. The data recovered from
practicing Vipassana meditation directly for prolonged periods of time will almost
certainly overlap with many such ideas but themes and narratives are likely
to be emergent, rather than conforming to an existing hypothesis of consciousness.

Phenomenology, in the broadest sense, may be used to describe experiences as they
pertain to a lifelong trajectory of altered traits. Phenomenological descriptions of
deep meditative states, or even of altered traits, is not the goal,
however. If other narrative tools emerge over the course of the study, I will augment
phenomenological descriptions and imagery with those tools.

Vipassana's instruction provides us with claims we can evaluate:
``[Vipassana] explores the entire field of mind and matter'' \cite{goenka1999discourses},
``[Vipassana] is a technique that will eradicate suffering'' \cite{dhammaorg2014code},
and ``Vipassana aims at ... total liberation and full enlightenment''
\cite{dhammaorg2014code}.
Progress made in evaluation of these claims will be, by its very nature, emergent.

The validity of these emergent observations made during the qualitative portion of
the study can be asserted across most axes of validity: prolonged engagement, rich
descriptions, external audits (from more experienced meditators), identification of
researcher bias, peer debriefing (again from more experienced meditators), and
searching for discrepancies in evidence are all possible. Respondent validation (member
checking), although possible, may not carry much weight in terms of strengthening the
validity of the study, given an initial sample of one. Triangulation will not be
possible with a sample of one. If the study proves productive, in the future
additional researcher-subjects intent on a lifelong Vipassana practice may strengthen
respondent validation and triangulation for parallel studies. Triangulation of
qualitative analysis of consciousness suffers from the paradox that any individual's
consciousness is only directly observable by that individual and conclusions --- even
those validated by triangulation --- will always be in the third-person.

Data will be collected on a daily basis and themes and narratives regularly collected
with the intention of
describing mental phenomena and the evolution of traits as Vipassana practice
progresses. Monthly or yearly schedules will be decided with a
study supervisor but an example daily schedule is available in
\textit{Appendix 1: Example Daily Schedule}.

\subsubsection{Quantitative Measures}

Measurements performed during the qualitative study will be used to create a
dataset upon which later inferences will be based.
Measurements taken throughout the study will be subject to equipment availability,
and as such will only be taken based on the future availability of the same
equipment.

In general, quantitative measurements will be taken by additional researchers,
as the study progresses.
This study is, in and of itself, not intended to interpret quantitative measures
of brain activity or otherwise.
Interpretation will be left to researchers familiar with such data, but raw
data will be publicized.

The simplest data available is from off-the-shelf EEG machines, available to
the general public with open-source hardware and open APIs, such as those
sold by BlueBCI \footnote{https://bluebci.com/}, making raw data
easily and permanently accessible to the study.
More advanced EEG machines will require partnership with a university or
research group in the future.

ECG data producing HR and HF-HRV data can be collected from off-the-shelf
systems such as the Zephyr Bioharness 3 \footnote{https://www.zephyranywhere.com/},
previously validated against physiology monitor systems used in laboratories.
\cite{johnstone2012bioharness}

fMRI scans are available at commercial
diagnostic laboratories in India for as little as \$100 USD per scan.
MEG, as a newer technology, will likely require a university partnership.

\subsection{Population and Study Sample}

Initially, only one researcher-subject will be used in this study.

Additional researcher-subjects will be added to the study in the future
if such persons are found capable of dedicating themselves to the task.

\subsection{Collection of Data}

Initial data collection will be done in the form of experiential reports.

\subsection{Data Management}

Data will be the intellectual property of Atapi Research Inc.

Data will be stored and backed up in raw formats produced by diagnostic
equipment under a version controlled data store.

\subsection{Participants}

Initially, the only participant in this study will by me (Steven Deobald).
In the future, additional researchers or research-subjects may be added
to the study.

\subsubsection{Why Steven Deobald?}

My willingness to submit myself to this study, for the length of time required,
is based on a number of factors.

First, I have been exclusively practicing Vipassana meditation for over 10 years.
I have taken month-long courses, and will commit myself to taking courses of
one or two months every year for the duration of the study.

Second, I am in the peculiar position of having severe eye damage due to a failed
surgery in 2014.
Although I can still use computers (and will need to do so to write experience
reports), it causes me quite a lot of physical harm to use computers for
extended periods of time.
A severe light sensitivity also caused by the failed surgery means that, peculiarly
enough, my eyes are most comfortable when they are closed in a dark room.

Third, I have been considering this form of research since 2016
--- nearly 8 years at the time of this writing.
I have also seriously considered monastic life, but ordaining as a monk is limiting
in terms of scientific output and there is no monastic order under the Vipassana
lineage yet.

Last, after 10+ years living in India, I am literate in the Devanagari script and
familiar with the Pāli language, which all original Buddhist texts are written in.
While researching meditation directly, it will be important to check findings
against the original teachings to validate my understanding before publishing it.

\subsection{Ethics and Human Subjects Issues}

The ethics of undertaking such a study are personal until and unless another
researcher-subject gets involved.
At that point, those individuals will need to assert for themselves that this
is definitely a process they want to undertake.

That said, there are some personal ethical constraints to address.

The study will leave little room for any other activity.
It is likely that other past-times will need to be nearly eliminated
to pursue the study in earnest.
``Weekends'', as a concept, won't really exist in a study which is intended
to operate continuously.

Other factors which are declared monastic rules, but which seem natural
while practicing deeper meditative states, may come into play.
Monastics are celibate and do not listen to music.
They are also not permitted to run.
While I do intend to become a celibate, eliminate physical exercise,
or stop listening to music, I am cognizant that these are risks of taking
a meditation practice serious to this degree.

\subsubsection{``Getting Paid to Meditate''}

On the subject of \textit{societal} ethics, the topic of
``getting paid to meditate'' has been one I'm acutely aware of, and one
that other commentators on the study have mentioned as well.

If the intention of the study were only to engage in the act of full-time
meditation, I would ordain as a monk.
At present, the system of rules laid out by most monasteries still reflect
those of the Vinaya: a minimum period of 5 years is required to live and
practice independently.
\footnote{https://www.watpahnanachat.org/joining}
Still, if I did not care to publish my investigations into Vipassana
meditation, five years of life is a small price to pay for the
independence to practice continuously.
\cite{nanatusita2014patimokkha}

However, the purpose of financing this study by sponsorship is to pay for
its outputs, not its inputs.
Sponsors who participate are justifiably expecting that the study will
produce readable outcomes, of which they will be the first recipients.

\subsection{Timeframes}

The nominal timeframe for the study will be throughout the course of the
life of the initial researcher-subject (me), and any further subjects
added to the study over time.

Results of the study and experience reports will be sent to sponsors
on a quarterly basis.
Less frequent aggregate results will be published in the form of
papers, articles, and books, with no fixed timeline.

\section{Strengths and Weaknesses of the Study}

The study will provide an in-depth examination of Vipassana meditation
as no other prior study has been able to formulate of any meditation
technique.
The sponsorship model will prevent the study from getting mired in
the examination of narrow outcomes.

As mentioned previously, the study is not structured to provide
significant statistical data out of which other researchers may draw
data-oriented conclusions.

\section{Risks}

There is a significant risk that such a broad study will devolve into
something of a performative art piece.
Since the writing produced by the study will be inherently subjective,
it will represent a sort of literary art form.
The line between Science and Art under these circumstances is blurry.
Avoiding this degredation of outputs is a primary concern.

There is a more acute risk that, after years of heavy meditation practice,
it will be natural to choose to renounce the householder life and take on
the role of a monk entirely, regardless of what other difficulities that
presents.

There is a final acute risk of financial unsustainability.
If the sponsorship model for the study cannot sustain even one person
financially, it will not be possible to continue with the study
indefinitely.

\section{Public Health Significance}

The significance of the study's outcomes for public health are parallel
to the current accelerating adoption of Vipassana meditation as the
primary meditation technique for many thousands, if not millions, of
householder and monastic practitioners.

If the outcomes of the study are resoundingly positive, it strongly
implies that Vipassana meditation could be an effective tool in
improving public health.
However, if negative effects are detected throughout the course of the
study, it could imply the opposite --- that we should be cautious in
employing Vipassana meditation as a tool for public health improvements.

\section{Budget}

Initially, the resources required to see this study through to completion
will be measured entirely by an individual's annual income.
To reduce the temptation of returning to a career in software development
(which I engaged in previously), that annual income would need to be
comparable to that of an intermediate software developer.
In Canada (where I currently reside), that figure is approximately
\$150,000 - \$200,000 CAD per annum.

As the study progresses, other resources will become a constraint, such as
access to commercial EEG, ECG, and fMRI equipment.
As such, the study will be run through Atapi Research Inc. to separate
study finances from my personal finances.

\section{Conflict of Interest Statement}

There is an inherent conflict of interest produced by the fact that I have
chosen Vipassana meditation as the single meditation technique I will use
for the rest of my life.
However, there is no way around this if the study is to focus on an
individual who has made precisely such a choice.

\section{Appendices}

\subsection{Appendix 1: Example Daily Schedule}

EXAMPLE: \\
05:00 - Wake \\
05:30 - 1 hour meditation \\
06:30 - Breakfast \\
07:30 - Exercise \\
09:00 - 3 hours meditation \\
12:00 - Lunch \\
13:00 - 3 hours meditation \\
16:00 - Writing period \\
18:00 - Evening meal \\
19:00 - 1 hour meditation \\
21:00 - Sleep \\

It is worth noting that this schedule generally persists across weekends and contains
flexibility for longer or shorter writing periods.
It is also worth noting that this is a sample schedule for the beginning of the study.
As the study progresses, it is expected that the duration of sleep will lessen.
Monthly and yearly schedules will include periods of total and partial seclusion,
such as meditation courses (24 hours of meditation per day, less deep sleep cycles)
and meditation retreats (16+ hours of meditation per day).

\section*{Acknowledgements}

Thank you to Preethi Govindarajan for reviews, edits, and corrections.

\newpage

% \section*{References}

\bibliography{the-proposal}{}
\bibliographystyle{chicago}

% \appendix*
% \input{sections/appendix1.tex}

\end{document}
