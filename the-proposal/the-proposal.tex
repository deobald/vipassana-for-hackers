\documentclass[a4paper, amsfonts, amssymb, amsmath, reprint, showkeys, nofootinbib, twoside]{revtex4-1}

\usepackage[english]{babel}
\usepackage[utf8]{inputenc}
\usepackage[colorinlistoftodos, color=green!40, prependcaption]{todonotes}
\usepackage{graphicx}
\usepackage{subcaption}
\usepackage{float}
\usepackage[bottom]{footmisc}
\usepackage{enumitem}
\usepackage{hyperref}

\setlist{noitemsep}

\begin{document}

\title{%
  \large{Vipassana for Hackers} \\
  \Huge{The Proposal} \\
  \large\textit{Version 0.1}
}
\author{Steven Deobald}
\email[Correspondence email address: ]{steven@deobald.ca}
\affiliation{www.vipassana-for-hackers.org}
\date{\today}

\begin{abstract}
Vipassana meditation (as taught by S.N. Goenka in the tradition of Sayagyi U Ba Khin)
is unlike other meditation techniques insofar as its claims to completeness and
outcomes are concerned. Vipassana claims to ultimately explore the entire field of
mind and matter, with the goals of total liberation and full enlightenment. Implicit
within these claims is a complete understanding of human consciousness. These are
difficult claims to prove because the time commitment required to research the
technique orders on multiple decades --- if not multiple generations. This is complicated by the
fact that the time commitment is demanded of both the researcher \textit{and the
subject}. Until now, research on meditation of all kinds has covered only one of
two fields: (1) controlled experiments which must inherently rely on superficial data
gathered from beginners and (2) observational exploratory research of monks --- expert
meditators who have dedicated their lives to the practice. I propose bridging this
gap by submitting myself to the middle ground. While remaining a layperson, I will
commit to a high ratio of waking medition hours for the rest of my life. Individually
and internally, I will conduct qualitative research into the consequences of
Vipassana meditation and the nature of consciousness while collectively and
externally pledging myself as a subject for long-term quantitative
studies with a broader community of researchers.

\todo{add references to (1) and (2)}

\end{abstract}

\keywords{neuroscience, psychology, vipassana, meditation}

\maketitle

\listoftodos

\section{Introduction}

Research into the effects of meditation has been conducted in earnest for half a
century but the quality of research in this field varies wildly. Randomized
controlled trials were missing from much research conducted during the first few
decades of meditation study. The importance of active controls was often missed even
when randomized controlled trials were attempted. Double-blind studies are inherently
impossible with meditation research; a subject will always know if she is receiving
meditation instruction or an active control instruction. \cite{goleman2017altered}

Add to these difficulties the very nature of meditation research itself. There are
many techniques of meditation and it is very important to capture the specific
technique under study to make meaningful assertions about its
effects. \cite{goleman2017altered} However, even within the definition of a single
meditation technique there exist variations in instruction between
teachers and each student's comprehension of the instructions
received. \cite{davidson2015conceptual} Even if researchers could cement (or at least
accurately record) semantics and terminology, the subject of study is often unclear:
Are we looking for health benefits? Increases in productivity? Increases in
intelligence? How long these effects persist? How much do we want to explore concrete
hypotheses versus exploratory analysis of long-term effects? How much can be learned
about the nature of consciousness? Can these learnings be measured objectively or
even communicated meaningfully?

Because Vipassana \footnote{For the remainder of this research proposal,
  ``Vipassana'' will always refer to the specific technique ``Vipassana as taught by S.N. Goenka in the
  tradition of Sayagyi U Ba Khin'' and lineage-identical instruction, such as that
  taught by Ledi Sayadaw (in writing) or other students of U Ba Khin contemporary
  to S.N. Goenka, unless noted otherwise.} is globally standardized, it affords
researchers with definitive
solutions to the difficulties presented by varying teaching methodologies. In
exchange for this, the difficulty of long-term study is compounded by the intrinsic
seriousness demanded of a Vipassana practitioner: the Pāli concept of \textit{ātāpī
  sampajāno satimā} (continuous piercing awareness of constantly
changing bodily sensation \cite{goenka1999discourses}) is not only a
requirement of serious Vipassana practice but could very well act as a surrogate
description of the practice itself. This seriousness poses obvious difficulties and
it is the intention of my study to begin breaking ground in solutions to those
difficulties.

This study will prove significant in three fields of research:

First, and most accessible, is the continued quantitative research of meditation in
the broader neuroscience and psychology disciplines, where my participation will be
more as subject than researcher.

Second is the qualitative research into the long-term consequences of
Vipassana meditation, what it reveals about the nature of human consciousness, and
reusable techniques for its exploration --- a field currently consisting of a bridge
between psychology, philosophy, linguistics, and contemplative studies.

Third is the exploratory research intended to objectively define
consciousness. As such research must pertain to all forms of consciousness it
therefore includes all non-human forms of consciousness. Findings will inform our
understanding of the Mind-Body Problem of psychology and philosophy, the entire field
of bioethics, and --- perhaps most importantly --- the nascent field of Artificial
Consciousness as a subfield of Artificial Intelligence. \cite{hildt2019artificial} As of this
writing, this overarching integral field of study has yet to emerge and has no name
as a discipline.

\section{Problem Statement}

\subsection{Overview}

\subsection{Hypothesis}

\section{Objectives and Aims}

\subsection{Overall Objective}
\subsection{Specific Aims}

\section{Background and Significance}

\subsection{Preliminary Research Review}
- summary of related research
- strengths and weaknesses
- justification
--- what hasn't been done by others?
--- why is this research necessary?

\subsection{Why Vipassana?}

1. ``complete'', standardized, global, multilingual, 100\% free
2. mundane (sleep) vs. supramundane (total eradication of suffering)



3. who is steven deobald

\subsection{Why now?}

\section{Research Design and Methods}

\subsection{Overview}

As the structure of studies conducted on Vipassana is inherently very difficult, due to
the strict nature of Vipassana meditation instruction, we must first examine which kinds of
studies are possible, which are not, and what will make possible studies worthwhile.

\subsubsection{Expertise}

A large scale controlled study of beginner-to-intermediate
students of Vipassana, each practicing a minimum of two hours daily, is
possible. Randomization will not be possible in such a study as the students
self-select this meditation technique for themselves. The recommendation to practice
the technique ``a minimum of two hours daily'' is a component of instruction, given
to students even on the most basic 10-day introductory
course. \cite{dhammaorg2017guidelines,goenka2001importance} As such, it is possible
to conduct such a study on students who have dedicated themselves to the practice of
Vipassana in the manner recommended --- and an increasingly large number of Vipassana
meditators do so, worldwide.

However, such a study suffers from the very nature of
the instruction and its target audience. On the lifelong scale of Vipassana
practice, the spectrum spans from a minimum of laypeople practicing two hours per day to
a maximum exemplified by renunciates (monks and nuns) who practice up to twenty-four
hours per day once they
reach the advanced stage where deep sleep no longer occurs. It is
also worth noting that a junior renunciate will still have less \textit{experience}
than a senior lay meditator, which means that the only objective measure of
experience is total number of hours meditated. \cite{goleman2017altered}

Meditation research often has difficulty defining and capturing \textit{expertise},
however, and total hours of experience is still confounded by the rate in which those
hours of experience are accumulated (hours per day). We must capture both, as accurately as
possible. Renunciates potentially have both a high hours-per-day rate and extended
(lifelong) duration of experience. The existence of renunciates forces the very
broad category of \textit{expert} meditators into the terrority of 100,000 hours
of practice, even if we limit practice hours to formal, sitting meditation and
estimate twelve (12) hours per day. If we take the much more modest rate of six (6)
hours per day, assume an adult renunciate can learn Vipassana, at the earliest, from
approximately 20 years of
age, and is now of an age when they may participate in an extended study (say, 50-60
years of age), we arrive at roughly 60,000 hours of practice. If we use the oft-cited
``10,000 hours'' measurement for competence in a subject, we might reasonably
describe our rough spectrum of expertise as such:

\begin{itemize}
	\item Beginner: 0-10,000 hours
	\item Intermediate: 10,000-60,000 hours
	\item Expert: 60,000+ hours
\end{itemize}

As a consequence, even studies which claim to observe ``expert'' or ``long-term''
Vipassana meditators are often predominantly observing beginners (7.9-8.6 mean years
of experience with 2 hours of daily practice). \cite{chiesa2010vipassana}

\subsubsection{Methodological Issues}

This is in addition to a large number of other
methodological issues with studies of Vipassana, as described by Alberto Chiesa in
\textit{Vipassana meditation: systematic review of current evidence},
2010. \cite{chiesa2010vipassana} These include the lack of: study replication,
randomized trials, active plus inactive controls, and double/single blinding.

Chiesa also notes that it would be beneficial to capture data ``both from a clinical
and from a neuro-imaging point of view'', including functional magnetic resonance
imaging (fMRI) and electroencephalography (EEG), for both short-term outcomes
(altered states) and long-term outcomes (altered traits \cite{davidson1977role}) to
improve future Vipassana research. \cite{chiesa2010vipassana}
Optically-pumped magnetometer (OPM)-based magnetoencephalography (MEG) performed using a
mobile helmet capable of operating at room temperature \cite{boto2017new}, in development since Chiesa's
paper was published, may permit it as a third neuro-imaging technique. Even older MEG
technology based on superconducting quantum interference devices (SQUIDs) could
potentially be used. The restriction of SQUID-based MEG is that subjects must remain
extremely still, but this is all but a requirement of serious Vipassana meditation
anyway.

Because Vipassana does not permit any form of imagination, any attempt to analyze it
through philosophical phenomenology becomes a hindrance to actually practicing
Vipassana. \cite{patrik1994phenomenological} The two are mututally exclusive, as is
any attempt to contemplate Vipassana, the phenomenology of the mind, or the
technique, during the actual practice of Vipassana. I have previously discussed this
apparent paradox in Appendix A of \textit{Vipassana for Hackers, Paper One: Curious Mechanics}.
\cite{deobald2017curious}

Last, statistical study of Vipassana, at the intermediate level, is currently not
possible. Meditators who are in the process of transitioning from 10,000 hours of
experience to 60,000 hours of experience are not readily available as research
subjects in significant numbers. Because we have not studied such meditators, we are
as of yet not exactly sure what we might be studying when we do study
them. V.S. Ramachandran put this most succinctly:

``I can't think of a single discovery of disease which had more than one initial
sample. .... You can't do statistical analysis of an initial discovery.''
\cite{ramachandran2019relevance}

\subsubsection{Meditation as Research Tool}

A lifelong study of a single, increasingly-experienced Vipassana meditator
is, in essence, an exploratory process of discovering what researchers might be
bothered to study in a (single) blind trial with active and inactive controls, over a
much larger sample. This process could be described as xxx

\todo{(Harari) / (Altered Traits) quotes}

Research tool initially, and research tool + research subject later on.

\subsubsection{Summary}

Because Vipassana meditators are inherently self-selecting, double blinding is not
possible, nor are randomized trials. Single blinding is possible, and should be
employed in future studies where it is feasible. Combination active/inactive controls
are also possible for all \textit{statistical} studies on Vipassana.


\subsection{Study Design / Research Method}
- case study type?

\subsection{POPULATION AND STUDY SAMPLE}
\subsection{SAMPLE SIZE AND SELECTION OF SAMPLE}
\subsection{SOURCES OF DATA}
\subsection{COLLECTION OF DATA}
\subsection{EXPOSURE ASSESSMENT}
\subsection{DATA MANAGEMENT}
\subsection{DATA ANALYSIS STRATEGIES}

\subsection{Participants}
- people and roles

\subsection{ETHICS AND HUMAN SUBJECTS ISSUES}
\subsection{TIMEFRAMES}

\section{STRENGTHS AND WEAKNESSES OF THE STUDY}
\section{PUBLIC HEALTH SIGNIFICANCE}
\section{BUDGET AND MOTIVATION}
- resources required

\section{Conflict of Interest Statement?}

\section{REFERENCES}
\section{APPENDICES}
\subsection{Appendix 1: Proposed Daily Schedule}

\section*{Acknowledgements}

Thank you to Preethi Govindarajan for reviews, edits, and corrections.


\section*{References}

\bibliography{the-proposal}{}
\bibliographystyle{chicago}

% \appendix*
% \input{sections/appendix1.tex}

\end{document}
