\documentclass[a4paper, amsfonts, amssymb, amsmath, reprint, showkeys, nofootinbib, twoside]{revtex4-1}

\usepackage[english]{babel}
\usepackage[utf8]{inputenc}
\usepackage[colorinlistoftodos, color=green!40, prependcaption]{todonotes}
\usepackage{graphicx}
\usepackage{subcaption}
\usepackage{float}
\usepackage[bottom]{footmisc}
\usepackage{enumitem}
\usepackage{hyperref}

\setlength{\parindent}{1.3em}
\setlength{\parskip}{0.7em}

\bibliographystyle{apsrev4-1}
\setlist{noitemsep}

\begin{document}


\title{%
  \large{Vipassana for Hackers} \\
  \Huge{Paper Four: Mastering Spacetime} \\
  \large\textit{Version 0.1}
}
\author{Steven Deobald}
\email[Correspondence email address: ]{steven@deobald.ca}
\affiliation{vipassana-for-hackers.org}
\date{\today}

\begin{abstract}
 \todo{After someone has completed their first 10-day Vipassana course there often
   remain a number of open questions regarding the practice. How, exactly, is the
   meditation removing mental negativities? Is this really the be-all-end-all of
   meditation techniques? Is the teaching methodology really systematic and global?
   This paper attempts to answer these questions and others.}
\end{abstract}

% \keywords{}

\maketitle

\section{Target Audience}

The target audience for this paper remains ``hackers'' for the broadest possible
definition of that word. \todo{add ref: Hackers book} If you are a curious and
analytical person, you qualify. The new restriction placed on this paper is that you
must have \textit{completed} at least one 10-day Vipassana course (as taught by
S.N. Goenka). To ``complete'' the course means that you left the course centre on the
eleventh day and no sooner. Although Sayagyi U Ba Khin authorized many other teachers
in addition to S.N. Goenka, the portions of this paper which deal with the systematic
and global approach to the teaching require taking a course from S.N. Goenka or one
of his authorized Assistant Teachers. Sayagyi U Ba Khin's other students did not
construct comparably systematic teaching methods.

This paper will discuss alternative methods of Vipassana instruction to S.N. Goenka's
system but even those discussions will be within the context of S.N. Goenka's
instruction as the global standard.

Readers may have chosen to dedicate their lives to the technique of Vipassana or they
may still be searching. Portions of this paper will be dedicated to analyzing reasons
to dedicate oneself to a single technique --- as well as merits and drawbacks of
choosing Vipassana as that technique.

The paper is best suited to Vipassana meditators who are looking to deepen and expand
their practice. It is important to remember, however, that this paper is authored by
non-teaching meditators and is no replacement for expert advice from a trained and
authorized Assistant Teacher.

With these restrictions in mind, anyone is of course welcome to read the
paper. However, much of what is being said will seem unnecessarily recondite to
someone without the requisite practical meditation experience.


\section{Eka Maggo: The One and Only Path}

\subsection{Learning Vipassana Elsewhere}

\begin{itemize}
  \item Ledi Sayadaw: 5 hours a day anapana, ``hairs of the head and the body''
  \item Other students of U Ba Khin: ``The Quiet Mind'', ``Beyond the Breath''
  \item ``Vipassana'': sensations matter more than terminology (3-day course, 10-day course), other forms of ``Vipassana'' not of interest, Zazen
\end{itemize}

\subsection{How is Vipassana ``Complete''?}

- 7 doors
- TM says to use your imagination exclusively, with a mantra
- Anapana variations say to use the breath (sensation) exclusively, with a limitation to a particular area
- Void-based meditations (yoga, elsewhere) suggest you can divorce yourself of attention from any one of these phenomena… which is pretty unlikely for a beginner
- “Sound of Silence” focuses on the ear sense door + tinitus
- other forms of Vipassana focus entirely on thoughts
- Zazen, generally, says open yourself to all 7 sense doors, all at once.
... they all choose an object of meditation within these 7 doors.

- mind / body === brain / nervous system

- vipassana the largest superset

- anapana covers time, vipassana covers space

Anapana eliminates space. Continuity is the source of constantly-narrowing attention on space.

It says “focus here, on this tiny area… make the area as tiny as possible, with no
limit or exception. Now all you have to do is follow your breath in the tiniest
slices possible, making time stretch out longer and longer with every sub-slice.”


Anapana is mastery over time.

Vipassana is harder. It stops ignoring space and actually adds it to the equation. You move your attention throughout your entire body (which is the only thing you have direct access to: sensation) with the narrow focus you trained yourself to achieve in the first 3 days of the course. As your apparent physical existence begins to break apart, you’ll open up your attention to the whole body, all at once (or as much as possible at once) but even after 9 Vipassana courses, I rarely ever do anything approximating that because I’m quite a beginner. 🙂

Anyway.

Vipassana is mastery over time + space.

...once you have that, there’s nothing left.

- Day 8: Meditate all the time.

- structure: the requirement that meditation be taught for free (expand)


\section{Concepts, Ideas, Help}

\begin{itemize}
  \item The Cheat Code: moving attention => sati vs. anicca
  \item Circle Awareness Drawing: how the 5 sense doors shut down
  \item Learn to sit cross-legged
\end{itemize}


\section{Warnings}

\begin{itemize}
  \item This document itself is a hindrance
  \item Don't get bogged down by ``perfect objectivity'' or ``perfect equanimity'' - yes muscles and blood flow follow your attention
  \item Do not use meditation to get high
  \item Do not mix meditations (1. dangers, 2. meditation ``doing itself'' - anatta)
\end{itemize}


\section*{Acknowledgements}

Thank you to Preethi Govindarajan for reviewing this paper.


\section*{References}

\begin{thebibliography}{99}

\bibitem{sense-icons}
  5 Senses by Daniel Falk from the Noun Project
  \url{https://thenounproject.com/daniel2021/collection/human-body-senses/}
  Thought by Nociconist from the Noun Project
  \url{https://thenounproject.com/search/?q=thought&i=2025873}
  Heart by Rafael Garcia Motta from the Noun Project
  \url{https://thenounproject.com/search/?q=heart&i=807960}
  Body by Makarenko Andrey from the Noun Project
  \url{https://thenounproject.com/search/?q=body&i=789989}
  \textit{The Noun Project}.

\bibitem{dhamma}
  Vipassana International Academy
  \url{https://www.dhamma.org}
  \textit{Vipassana Meditation Website}


\end{thebibliography}


\end{document}
