\documentclass[a4paper, amsfonts, amssymb, amsmath, reprint, showkeys, nofootinbib, twoside]{revtex4-1}

\usepackage[english,hindi]{babel}
\usepackage[utf8]{inputenc}
\usepackage[colorinlistoftodos, color=green!40, prependcaption]{todonotes}
\usepackage{graphicx}
\usepackage{subcaption}
\usepackage{float}
\usepackage[bottom]{footmisc}
\usepackage{enumitem}
\usepackage{hyperref}

\bibliographystyle{apsrev4-1}
\setlist{noitemsep}

\begin{document}


\title{%
  \large{Vipassana for Hackers} \\
  \Huge{Paper Four: Mastering Spacetime} \\
  \large\textit{Version 0.1}
}
\author{Steven Deobald}
\email[Correspondence email address: ]{steven@deobald.ca}
\affiliation{vipassana-for-hackers.org}
\date{\today}

\begin{abstract}
 \todo{After someone has completed their first 10-day Vipassana course there often
   remain a number of open questions regarding the practice. How, exactly, is the
   meditation removing mental negativities? Is this really the be-all-end-all of
   meditation techniques? Is the teaching methodology really systematic and global?
   This paper attempts to answer these questions and others.}
\end{abstract}

% \keywords{}

\maketitle

\section{Target Audience}

The target audience for this paper remains ``hackers'' for the broadest possible
definition of that word. \todo{add ref: Hackers book} If you are a curious and
analytical person, you qualify. The new restriction placed on this paper is that you
must have \textit{completed} at least one 10-day Vipassana course (as taught by
S.N. Goenka). To ``complete'' the course means that you left the course centre on the
eleventh day and no sooner. Although Sayagyi U Ba Khin authorized many other teachers
in addition to S.N. Goenka, the portions of this paper which deal with the systematic
and global approach to the teaching require taking a course from S.N. Goenka or one
of his authorized Assistant Teachers. Sayagyi U Ba Khin's other students did not
construct comparably systematic teaching methods.

This paper will discuss alternative methods of Vipassana instruction to S.N. Goenka's
system but even those discussions will be within the context of S.N. Goenka's
instruction as the global standard.

Readers may have chosen to dedicate their lives to the technique of Vipassana or they
may still be searching. Portions of this paper will be dedicated to analyzing reasons
to dedicate oneself to a single technique --- as well as merits and drawbacks of
choosing Vipassana as that technique.

The paper is best suited to Vipassana meditators who are looking to deepen and expand
their practice. It is important to remember, however, that this paper is authored by
non-teaching meditators and is no replacement for expert advice from a trained and
authorized Assistant Teacher.

With these restrictions in mind, anyone is of course welcome to read the
paper. However, much of what is being said will seem unnecessarily recondite to
someone without the requisite practical meditation experience.


\section{Eka Maggo: The One and Only Path}

From one's very first Vipassana course --- in fact, from the discourse on the second
day --- a meditator is met with two messages: that of \textit{Eka Maggo} (the one and
only path out of suffering) and the idea that the Noble Eightfold Path is
``complete''. This section will address these two issues.

\subsection{Learning Vipassana Elsewhere}

For new meditators, Goenkaji addresses this issue during early
courses. Although it isn't a point that is repeatedly addressed, there are two quotes
worth highlighting in this regard:

\begin{quote}
  \textbf{You call it ``Vipassana'' --- you call it whatever you like.}
\end{quote}

\todo{add ref to quote}

\begin{quote}
  \textbf{Any technique which deals with sensations on the body is a technique of
  liberation.}
\end{quote}

\todo{add ref to quote}

Dissecting these two ideas quickly brings us to the conclusion that Goenka's
instruction in no way implies that he is the only legitimate teacher of Vipassana ---
just the most accessible. Anyone who has ever attempted a 10-day self course will
attest to the sheer difficulty of attempting to recreate the environment of a
Vipassana Centre in one's own home. But it can be done and is even recommended in the
Day 11 discourse before the course ends. \todo{add reference to discourse}

Between the 10-day introductory course, the 3-day course, and the Satipatthana Sutta
course, the messaging becomes very clear: Vedana (sensations) are of utmost
importance and terminology is of no importance. Many techniques refer to themselves
as ``Vipassana'' but if they do not deal with sensation they are not relevant to the
discussion, as we will see below in \textit{How is Vipassana ``Complete''?} Other
techniques, such as Zazen and Kundalini, which deal with sensation at least in part
are of interest to a meditator convinced of the value of sensation analysis. It is up
to each meditator to decide whether they prefer one technique or the other for
analyzing sensation and pursue that one technique vigorously. The reasons for
choosing only a single technique will be discussed in \textit{Warnings: Do not mix
  meditations}.

If a meditator finds herself in the position of preferring Vipassana, with an
emphasis on Vedana and Anicca, but preferring, for one reason or another, not to
learn the technique from S.N. Goenka and his Assistant Teachers, there are other
methods for learning within the same lineage.

Going back one generation, it is possible to learn from the other students of Sayagyi
U Ba Khin. Though most of them are getting quite old, some do still teach. John
Coleman describes teaching his first 10-day course in the UK in \textit{The Quiet
  Mind} \todo{add ref}, for example. Glickman, another student of Sayagyi U Ba Khin,
attempts another strategy by laying out a progressive and systematic at-home training
plan for readers in his book \textit{Beyond The Breath: Full Body Vipassana
  Meditation} \todo{add ref}.

We cannot go back two generations as Saya Thet Gyi did not leave any written
instructional material, nor any known authorized teachers other than his student,
Sayagyi U Ba Khin. However, we can go back three generations to Saya Thet's teacher:
Ledi Sayadaw. Ledi Sayadaw was both a master meditator and accomplished scholar and
wrote prolifically in Burmese and Pali. Although his work is often deeply arcane, in
his \textit{Manuals of Dhamma} and \textit{Anapana Dipani} \todo{add ref} he
specifies his instructions for practicing Anapana and Vipassana as a layperson:


\begin{quote}
  A person who wants to practise vipassanā, being an ordinary human being,
    may not find it possible to put forth effort twenty-four hours a day. He must
    therefore allocate three or four hours a day and put
    forth effort punctually and regularly every day. When he starts to practise, he must
    first overcome the wandering tendencies of the mind and establish mindfulness on the
    breath. It is only after he has overcome the mind’s wandering tendencies that he can
    direct the mind towards vipassanā.

    ....

    The way to rid oneself of [the belief that is firmly and deeply rooted in the minds
      of worldlings (puthujjana)] is as follows: When the eye of wisdom penetrates to
    these four primary elements and the ultimate reality is perceived, such things as
    shape and form in the out-breath and in-breath disappear, and every time one
    contemplates them, the deep and firm root of personality view disappears. One
    perceives that there is in reality no shape and form --- no out-breath and
    in-breath. One perceives that there exist only the four primary elements. Thus
    purity of view (diṭṭhi-visuddhi) is achieved.

    ....

    It is the same with respect to the other parts of the body such as head-hairs,
    body-hairs, etc. There exists, on the one hand, the deeply rooted habitual
    perception of shape and form, such as, ``This is head-hair,'' and on the other, there
    exist the four primary elements. When these four primary elements are penetrated
    and clearly perceived with wisdom in the head-hairs, the deeply rooted wrong
    perception of shape and form will disappear. It will be perceived that the
    head-hairs do not exist in reality. When it is thus seen, purity of view in the
    head-hairs is achieved. Proceed in the same way in the case of the other parts of
    the body such as body-hairs, etc.
\end{quote}

Ledi Sayadaw's recommendations in \textit{The Manuals of Dhamma} are similar. To
parapharse: Practice Anapanasati by observing the in-breath and the out-breath at the
``point of touch'' above the upper lip for 4 to 5 hours a day for a few months. Then
investigate the hairs of the head and the hairs of the body.

One assumes subtle sensation would be fairly easy to detect if each individual hair
is easily visible to the attention at the \textit{beginning} of one's Vipassana
practice.

\subsection{How is Vipassana ``Complete''?}

There is a finite set of possible destinations for our attention, a finite set of
possible meditation objects. These can be
enumerated in various ways but all enumerations tend to be equivalent to the six
sense doors of traditional Buddhist theory (eyes, ears, tongue, nose, body skin, and mind) plus
the seventh sense door of Vedana
(sensation) inside the body. Some Westernized variations of this enumeration choose
to divide the sixth sense door of \textit{mind} into two categories of
\textit{thought} and \textit{emotion}, however, this division will provide a
limitation in deeper meditative states where states of mind no longer resemble either
of these two categories. Rather than devise new terminology, ever dividing mind
states into narrower and more specific subcategories, it makes the most sense to
simply leave the sixth sense door of \textit{mind} intact.

\todo{add 8-sense diagram}

The other common
complication of categorization is that of \textit{body and body skin}, which is ultimately the sense
door at which Vedana arise. Even within the scope of a 10-day Vipassana course,
sensation is first addressed where it is most accessible: the body skin. Gross,
solidified Vedana (say, internal sensation such as
a stomach ache or external sensation such as a feather on the skin) can be thought of
as the sense object and subtle Vedana (perhaps elsewhere on the body) can be thought of as the corresponding Vedana
understood to be addressed in observation of impermanence (Anicca and Sampajañña
\todo{add ref to importance of v and s}). This oversimplification will eventually prove insufficient, however,
as the sense-door bodily sensations and Vedana of Sampajañña are recursively
codependent --- you cannot have one without the other. Ultimately, they are ``two
sides of the same coin'' \todo{add ref to discourse 10}.

\todo{add sense reflection diagram}

Having the experience of at least one 10-day Vipassana course, one understands ---
experientially --- the relationship of sense door contact to sensation on the
body. That is, when contact is made at any sense door there is a sensation on the
body and it is possible to observe this sensation, objectively. In meditation, the
most important sense door tends to be the mind. The meditator may go to a quiet,
uninhabited place, sit down cross-legged, and proceed to close her eyes and
mouth. It is easy to physically stopple the external sense doors. But the sense door
of the mind may only become louder on such an occasion. The sense door of the mind is
also reflective of the other five senses: it remembers and anticipates contact at the
other five sense doors, in addition to its own unique processes of automatic
verbalization and emotional habit. Regardless of which sense door is activated, mind
or otherwise, a reflective sensation occurs within the body. This is not a point
which can be explained in text or conveyed from person to person. It must absolutely
be experienced by the individual.

With both this reflective property of Vedana and the finite set of meditation
objects in mind, let us examine some other meditation techniques.

\textbf{Transcendental Meditation (TM)} is a popularization of mantra meditation,
found in various forms of Yoga and elsewhere. It employs a system of bija
mantras \todo{add reference to bija mantra list}, silently recited. In terms of sense
doors, five are stoppled, the mind sense door is employed with a remembered object of
the sound sense door. Bodily sensation is ignored.

\textbf{Anapanasati} is the broad spectrum of meditations which employ the in-breath
and out-breath as the object of meditation. The breath can only be \textit{felt} and
therefore relies on either body-skin sensation or other Vedana. There are variations
which place the attention at various points within the body: the tip of the nose, the
upper lip, the passage of breath down to the diaphragm or navel, exclusively on
the stomach, etc. These all demand the body as meditation object and thus rely on
sensation as the door of perception for meditation but, in general, variations of
Anapanasati do not extend to the \textit{entire} body. As some sensation on portions
of the body irrelevant to the in-breath and out-breath are ignored, such meditation
techniques can never be considered complete in the sense that they do not and cannot
encompass the entirety of sensory experience. Exceptions do exist. Most notably,
Anapanasati instruction by Webu Sayadaw, which directs attention to the ``point of
touch'' above the upper lip but admits that very narrow Anapanasati, when continued
for 24 hours per day at length, will eventually lead to awareness of sensation across
the entire body.

\textbf{Void (Emptiness) Meditations} found in some Yogic and Zen schools and
elsewhere suggest that the meditator can
divorce herself from attention entirely, removing attention from any one of these
phenomena and placing it on nothing. Such a technique is unlikely to succeed for
beginners and the path provided in Void Meditation teachings inherently begins with a
mental ``visualization'' at any one of the six sense doors: an absence of sight
object, sound object, feeling object, etc. Where an apparent void is possible to
observe directly, a meditator may have attained access concentration \todo{add ref to
  access concentration} or a jhana \todo{add ref to jhana}. However, attention
directed toward a minimum (or zero, if that is possible) sense doors is inherently
attention which misses something. By its very nature, it is attention which tries to
miss as many inputs as possible. Such a meditation cannot be considered complete,
given the vast surface area of sense contact which is intentionally missed.

\textbf{Sound of Silence / Nada Yoga} is a void-adjacent meditation technique taught
by Ajahn Amaro and others which, paradoxically, narrows attention specifically to the
ear sense door. For most people ``the sound of silence'' is really the sound of
tinitus \todo{add reference to tinitus} as the human ear is largely incapable of
hearing nothing at all. As Anapanasati may eventually lead to full-body Sampajañña,
so may the Sound of Silence. Sayagyi U Ba Khin describes access to full-body Vedana
by means of the other five sense doors (other than the body skin) but insists that
the body skin is by far the easiest and most accessible to lay meditators. \todo{add
  ref to u ba khin discourse}

\textbf{Vipassana (Cittanupassana)} \todo{add ref to cittanupassana} as taught by
Mahasi Sayadaw and others is perhaps
even less accessible than the Sound of Silence as an alternative sense door to
internal Vedana. Cittanupassana, taught without Vedana as the meditation object,
tends to embroil the meditator in the objects of gross, surface-level thought and
emotional processes. Although it is theoretically possible for a meditator to work
with sensation-based Sampajañña through this method, it would seem extremely unlikely
for anyone but a monk or nun to do so.

\textbf{Zazen} generally suggests that the meditator open herself to all seven
(counting Vedana of the body and body-skin as separate) sense doors,
simultaneously. In this sense, Zazen can be thought of as ``complete'' but from
research of lay meditators it seems extremely difficult for the meditator to practice
to a point where subtle sensation is felt on or inside the body.

\textbf{Vipassana} asks the meditator, initially, to stopple the five external sense
doors. Sit in a shunyagar (empty room or cell), in reduced light, in a quiet and uninhabited
environment, with eyes and mouth closed. Since the mind is active, attention is
repeatedly returned to bodily sensation: first at the narrowest possible location
under the nostrils during the Anapana period and second at the widest possible set of
locations --- across the entire body --- during the Vipassana period. As the six
sense bases are fully reflected in bodily sensation, Vipassana on full-body sensation
will (eventually) cover the entire field of sensory, mental, and supramundane
experience. An explanation is provided in the context of Spacetime, below.

\subsection{Mastering Spacetime}



- mind / body === brain / nervous system

- vipassana the largest superset

- anapana covers time, vipassana covers space

Anapana eliminates space. Continuity is the source of constantly-narrowing attention on space.

It says “focus here, on this tiny area… make the area as tiny as possible, with no
limit or exception. Now all you have to do is follow your breath in the tiniest
slices possible, making time stretch out longer and longer with every sub-slice.”


Anapana is mastery over time.

Vipassana is harder. It stops ignoring space and actually adds it to the equation. You move your attention throughout your entire body (which is the only thing you have direct access to: sensation) with the narrow focus you trained yourself to achieve in the first 3 days of the course. As your apparent physical existence begins to break apart, you’ll open up your attention to the whole body, all at once (or as much as possible at once) but even after 9 Vipassana courses, I rarely ever do anything approximating that because I’m quite a beginner. 🙂

Anyway.

Vipassana is mastery over time + space.

...once you have that, there’s nothing left.

- Day 8: Meditate all the time.

\section{structure: the requirement that meditation be taught for free (expand)}


\section{Concepts, Ideas, Help}

\begin{itemize}
  \item The Cheat Code: moving attention => sati vs. anicca
  \item Circle Awareness Drawing: how the 5 sense doors shut down
  \item Learn to sit cross-legged
\end{itemize}


\section{Warnings}

\begin{itemize}
  \item This document itself is a hindrance
  \item Don't get bogged down by ``perfect objectivity'' or ``perfect equanimity'' - yes muscles and blood flow follow your attention
  \item Do not use meditation to get high
\end{itemize}

\subsection{Do not mix meditations}

(1. dangers, 2. meditation ``doing itself'' - anatta)


\section*{Acknowledgements}

Thank you to Preethi Govindarajan for reviewing this paper.


\section*{References}

\begin{thebibliography}{99}

\bibitem{sense-icons}
  5 Senses by Daniel Falk from the Noun Project
  \url{https://thenounproject.com/daniel2021/collection/human-body-senses/}
  Thought by Nociconist from the Noun Project
  \url{https://thenounproject.com/search/?q=thought&i=2025873}
  Heart by Rafael Garcia Motta from the Noun Project
  \url{https://thenounproject.com/search/?q=heart&i=807960}
  Body by Makarenko Andrey from the Noun Project
  \url{https://thenounproject.com/search/?q=body&i=789989}
  \textit{The Noun Project}.

\bibitem{dhamma}
  Vipassana International Academy
  \url{https://www.dhamma.org}
  \textit{Vipassana Meditation Website}


\end{thebibliography}


\end{document}
